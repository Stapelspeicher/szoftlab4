
\subsection{GameController}
\begin{itemize}
	\item Felelősség\newline
	Létrehozza a játékot, az egész játék irányításáért felelős osztály. A generált pályához létrehoz annyi robotot, ahányat az elején kiválasztanak a felhasználók. A kör léptetése is itt történik.
	
	\item Attribútumok

	\begin{itemize}
		\item -mainWindow: Window: Az ablakot megjelenítő objektum
		\item -game: GameMap: Az aktuális játékmenet pályája
		\item -gameStartPanel: GameStartPanel: A játék indításához szükséges panel
		\item -resultPanel: ResultPanel: Az eredményjelző panel.
		\item -players: int: A játékosok száma.
		\item -robots[]: Robot: A robotokat tároló objektum.
		\item -littleRobots: ArrayList<LittleRobots>: A kisrobotokat tároló objektum. 
		\item -randomGenerator: Random: Random generátor a robotok és kisrobotok helyzetének inicializálásához.
		\item -currentRobot: int: Az épp aktuálisan kijelölt robot.
		\item -robotsAlive: int: Az életben lévő robotok száma.
		\item -gamePanel: GamePanel: A játékteret megjelenítő panel
		\item \underline{+GAMEMAPSIZE: int}: A pálya méretét eltároló konstans. (cellák száma x és y irányban)
		\item \underline{-OILY: int}: A robotok rendelkezésére álló olajfoltok száma.
		\item \underline{-STICKY: int}: A robotok rendelkezésére álló ragacsfoltok száma.
		\end{itemize}
	\item Metódusok
	\begin{itemize}
		\item +startGame(int players): void: Létrehozza a pályát, beállítja a robotok és a kisrobotok számát. 
		\item +selectPlayers(): void: A grafikus felület játékosválasztáshoz szükséges nézetét jeleníti meg.
		\item -endGame(): void: A játék végét jelző eseményt hajtja végre.
		\item -nextRound(): void: Csökkenti a körök számát és lépteti a kisrobotokat, valamint létrehoz egy újabb kisrobotot minden 3. körben.
		\item +placeOily(): void: Olajfolt elhelyezése az aktuális robottal. 
		\item +placeSticky(): void: Ragacsfolt elhelyezése az aktuális robottal.
		\item +robotStep(int x, int y): Lépteti az aktuális robotot.
		\item \underline{+main(String args[]): void:} A játék Main függvénye, csak inicializál.
	\end{itemize}
\end{itemize}

\subsection{GameMapView}
\begin{itemize}
	\item Felelősség\newline
	A pályát kirajzoló osztály. Kirajzolja a négyzetrácsos területet, ezen a ragacsokat és az oljafoltokat.

	\item Attribútumok
	\begin{itemize}
		\item game: GameMap: 
		\item \underline{+BORDERWIDTH: int}:
		\item \underline{+CELLSIZE: int}:
		\item \underline{+CELLGAP: int}:
	\end{itemize}
	\item Metódusok
	\begin{itemize}
		\item +GameMapView(GameMap g): Az osztály konstruktora.
		\item +paintMap(Graphics g): void: A pályát rajzolja ki, amelyen majd a játék zajlani fog.
		\item -paintOily(Graphics g, int x, int y): void: Kirajzolja az olajfoltokat
		\item -paintSticky(Graphics g, int x, int y): void: Kirajzolja a ragacsfoltokat
		\item +paintTraps(Graphics g): void: A csapdákat kirajzoló függvény, meghívja a paintOily és paintSticky metódusokat.
	\end{itemize}
\end{itemize}

\subsection{RobotView}
\begin{itemize}
	\item Felelősség\newline
	A robotok megjelenítése a felelőssége.
	
	\item Attribútumok
	\begin{itemize}
		\item \underline{+robotColors[]: Color}: A robotok színeit tartalmazza, konstans értékek. 
		\item -robot: Robot: A kirajzolandó robot objektum.
		\item -colorIndex: int: A robothoz rendelt szín indexe
	\end{itemize}
	\item Metódusok
	\begin{itemize}
		\item +RobotView(Robot r, int i): Az osztály konstruktora, hozzárendeli a robothoz a kapott színt.
		\item +paint(Graphics g): void: Kirajzolja a robotot.
	\end{itemize}
\end{itemize}

\subsection{LittleRobotView}
\begin{itemize}
	\item Felelősség\newline
	A kisrobotok megjelenítése a felelőssége.

	\item Attribútumok
	\begin{itemize}
		\item littleRobot: LittleRobot: A kirajzolandó kisrobot objektum.
		
	\end{itemize}
	\item Metódusok
	\begin{itemize}
		\item +LittleRobotView(LittleRobot lr): Az osztály konstruktora, beállítja a kirajzolandó kisrobotot.
		\item +paint(Graphics g): void: Kirajzolja a kisrobotot.
	\end{itemize}
\end{itemize}

\subsection{Window}
\begin{itemize}
	\item Felelősség\newline
	Az egész felület containere, egy jobboldali és egy baloldali panelből áll.
	\item Ősosztályok\newline

		JFrame $\rightarrow$ Window
		
	\item Attribútumok
	\begin{itemize}
		\item \underline{+WINDOWSIZEX: int}: Az ablak szélessége, konstans
		\item \underline{+WINDOWSIZEY: int}: Az ablak magassága, konstans
		\item -leftPanel: JPanel: Baloldali panel.
		\item -rightPanel: JPanel: Jobboldali panel.
		\item -resultPanel: JPanel: Az eredményeket megjelenítő panel.
		\item -controllerPanel: JPanel: Az irányító panel.
	\end{itemize}
	\item Metódusok
	\begin{itemize}
		\item +Window(): Az osztály konstruktora. Beállítja az ablak méretét és az elrendezését
		\item +setLeftPanel(JPanel panel): void: Beállítja a baloldali panelt, kirajzolja az ablak bal oldalára.
		\item +setResultPanel(JPanel panel): void: Beállítja az eredmény panelt, a jobb oldali panel felső részén helyezi el.
		\item +setControllerPanel(JPanel panel): void: Az irányító panelt állítja be, a jobboldali panel aljára helyezi el.
	\end{itemize}
\end{itemize}

\subsection{GamePanel}
\begin{itemize}
	\item Felelősség\newline
	Egy aktív játékteret ábrázoló panel, A Window osztály container bal oldala. A modell megváltozása (user input hatására) után a GamePanel mindig újrarajzolódik, s meghívja a kirajzoló objektumok megfelelő kirajzoló metódusait.
	\item Ősosztályok\newline
		JPanel $\rightarrow$ GamePanel
	\item Attribútumok
	\begin{itemize}
		\item -gameMap: GameMapView: A kirajzolandó pálya grafikus objektuma
		\item -robots[]: RobotView: A robotok grafikus objektumai
		\item -littleRobots: ArrayList<LittleRobotView>: A kisorobotok grafikus objektumai
	\end{itemize}
	\item Metódusok
	\begin{itemize}
		\item +GamePanel(): A GamePanel konstruktora, beállítja a panel méretét és hátterét.
		\item +setGame(GameMap g): void: Beállítja a kirajzolandó játékteret
		\item +setRobots(Robot robots[]): Létrehozza a robotok grafikus objektumait, hozzáadja a megfelelő listához, majd kirajzolja őket.
		\item +addLittleRobot(LittleRobot lr): void: Létrehozza a kisrobot grafikus objektumát, hozzáadja a megfelelő listához, majd kirajzolja.
		\item +paintComponent(Graphics g): void: Meghívja a komponenseket (robot, kisrobot, csapdák) megrajzoló függvényeket.
	\end{itemize}
\end{itemize}

\subsection{ResultPanel}
\begin{itemize}
	\item Felelősség
	Az eredményeket eltároló panel. Megjeleníti a a robotokhoz tartozó távolságokat, valamint az éppen aktív robot sebességét és foltkészletét.
	\item Ősosztályok\newline
		JPanel $\rightarrow$ ResultPanel
		
		\item Metódusok
		\begin{itemize}
			\item +ResultPanel(Robot robots[]): Az osztály konstruktora. Beállítja a panel hátterét és elrendezését, majd a panelhez rendeli a robotok eredményeit, amelyek külön panelek.
		\end{itemize}
	\end{itemize}
	
\subsection{RobotResultPanel}
\begin{itemize}
	\item Felelősség\newline
	Egy adott robot eredményét megjelenítő panel.
	\item Ősosztályok\newline

		JPanel $\rightarrow$ RobotResultPanel
		
		\item Attribútumok
		\begin{itemize}
			\item -robot: Robot: A robot objektum, aminek az eredménye megjelenítendő
			\item -distancePanel: JPanel: A távolságot megjelenítő panel.
			\item -xPanel: JPanel: A vízszintes sebességet megjelenítő panel
			\item -yPanel: JPanel: A függőleges sebességet megjelenítő panel
			\item -oilyPanel: JPanel: Az robot rendelkezésére álló olajfoltok számát megjelenítő panel.
			\item -stickyPanel: JPanel: Az robot rendelkezésére álló olajfoltok számát megjelenítő panel.
			\item distanceValueLabel: JLabel: A megtett távolságot tartalmazó szöveg.
		\end{itemize}
		\item Metódusok
		\begin{itemize}
			\item +RobotResultPanel(Robot r, int i): Az osztály konstruktora. Beállítja az elrendezést, és a robotoknak megfelelő színeket, valamint létrehozza a paneleket
			\item +paintComponent(Graphics g): void: Frissíti a megjelenítést.
		\end{itemize}
	\end{itemize}
	
\subsection{SelectorButton}
\begin{itemize}
	\item Felelősség\newline
	\item Ősosztályok\newline
		JButton $\rightarrow$ SelectorButton
	\item Attribútumok
	\begin{itemize}
		\item -id: int: A gomb azonosítója (hányadik a sorban)
	\end{itemize}
	\item Metódusok
	\begin{itemize}
		\item +SelectorButton(int id): Az osztály kontsruktora.
		\item +getID: int: Visszaadja az azonosítót
	\end{itemize}
\end{itemize}






