% Szglab4
% ===========================================================================
%
\chapter{Grafikus felület specifikációja}

\thispagestyle{fancy}

\section{Fordítási és futtatási útmutató}
\comment{A feltöltött program fordításával és futtatásával kapcsolatos útmutatás. Ennek tartalmaznia kell leltárszerűen az egyes fájlok pontos nevét, méretét byte-ban, keletkezési idejét, valamint azt, hogy a fájlban mi került megvalósításra.}

\subsection{Fájllista}

\begin{fajllista}

\begin{fajllista}

\fajl
{ActiveObject.java} % Kezdet
{1588 byte} % Idptartam
{2015.03.21~23:00~} % Résztvevők
{A játék mozgó résztvevőinek a közös interface} % Leírás

\fajl
{Cell.java} % Kezdet
{5025 byte} % Idptartam
{2015.03.21~23:00~} % Résztvevők
{Az cellákat reprezentáló osztály} % Leírás

\fajl
{GameMap.java} % Kezdet
{3438 byte} % Idptartam
{2015.03.21~23:00~} % Résztvevők
{A játékteret reprezentáló osztály} % Leírás

\fajl
{LittleRobot.java}
{3313 byte}
{2015.04.10~20:07~}
{A kisrobotokat megvalósító osztály}

\fajl
{LittleRobotState.java}
{96 byte}
{2015.04.10~20:07~}
{A kisrobotok állapotait megvalósító enumeráció}

\fajl
{Logger.java} % Kezdet
{1032 byte} % Idptartam
{2015.03.21~23:00~} % Résztvevők
{A függvényhívások konzolra Loggolásához használt osztály} % Leírás

\fajl
{Oily.java} % Kezdet
{761 byte} % Idptartam
{2015.03.21~23:00~} % Résztvevők
{Az olajfoltot reprezentáló osztály} % Leírás

\fajl
{Position.java} % Kezdet
{2661 byte} % Idptartam
{2015.03.21~23:00~} % Résztvevők
{A játéktér koordinátáit reprezentáló osztály} % Leírás

\fajl
{Robot.java} % Kezdet
{4836 byte} % Idptartam
{2015.03.21~23:00~} % Résztvevők
{A Robotokat megvalósító osztály} % Leírás

\fajl
{Sticky.java} % Kezdet
{809 byte} % Idptartam
{2015.03.21~23:00~} % Résztvevők
{A Ragacs csapdát megvalósító osztály} % Leírás

\fajl
{Trap.java} % Kezdet
{706 byte} % Idptartam
{2015.03.21~23:00~} % Résztvevők
{A csapdák közös interface} % Leírás

\end{fajllista}

\subsection{Fordítás}
\comment{A fenti listában szereplő forrásfájlokból milyen műveletekkel lehet a bináris, futtatható kódot előállítani. Az előállításhoz csak a 2. Követelmények c. dokumentumban leírt környezetet szabad előírni.}

\lstset{escapeinside=`', xleftmargin=10pt, frame=single, basicstyle=\ttfamily\footnotesize, language=sh}
\begin{lstlisting}
javac -d bin *.java
\end{lstlisting}

\subsection{Futtatás}
\comment{A futtatható kód elindításával kapcsolatos teendők leírása. Az indításhoz csak a 2. Követelmények c. dokumentumban leírt környezetet szabad előírni.}

\lstset{escapeinside=`', xleftmargin=10pt, frame=single, basicstyle=\ttfamily\footnotesize, language=sh}
\begin{lstlisting}
cd bin
java Main.java
\end{lstlisting}

\section{Értékelés}
\comment{A projekt kezdete óta az értékelésig eltelt időben tagokra bontva, százalékban.}

\begin{ertekeles}
\tag{Horváth} % Tag neve
{23.5}        % Munka szazalekban
\tag{Német}
{24.5}
\tag{Tóth}
{25}
\tag{Oláh}
{27}
\end{ertekeles}

