% Szglab4
% ===========================================================================
%
\chapter{Szkeleton beadás}

\thispagestyle{fancy}

\section{Fordítási és futtatási útmutató}

\subsection{Fájllista}

\begin{fajllista}

\fajl
{ActiveObject.java} % Kezdet
{256 byte} % Idptartam
{2015.03.21~23:00~} % Résztvevők
{A játék mozgó résztvevőinek a közös interface} % Leírás

\fajl
{Cell.java} % Kezdet
{2002 byte} % Idptartam
{2015.03.21~23:00~} % Résztvevők
{Az cellákat reprezentáló osztály} % Leírás

\fajl
{GameMap.java} % Kezdet
{2255 byte} % Idptartam
{2015.03.21~23:00~} % Résztvevők
{A játékteret reprezentáló osztály} % Leírás

\fajl
{Logger.java} % Kezdet
{961 byte} % Idptartam
{2015.03.21~23:00~} % Résztvevők
{A függvényhívások konzolra Loggolásához használt osztály} % Leírás

\fajl
{Oily.java} % Kezdet
{242 byte} % Idptartam
{2015.03.21~23:00~} % Résztvevők
{Az olajfoltot reprezentáló osztály} % Leírás

\fajl
{Position.java} % Kezdet
{970 byte} % Idptartam
{2015.03.21~23:00~} % Résztvevők
{A játéktér koordinátáit reprezentáló osztály} % Leírás

\fajl
{Robot.java} % Kezdet
{2505 byte} % Idptartam
{2015.03.21~23:00~} % Résztvevők
{A Robotokat megvalósító osztály} % Leírás

\fajl
{Sticky.java} % Kezdet
{284 byte} % Idptartam
{2015.03.21~23:00~} % Résztvevők
{A Ragacs csapdát megvalósító osztály} % Leírás

\fajl
{Test.java} % Kezdet
{8090 byte} % Idptartam
{2015.03.21~23:00~} % Résztvevők
{A teszteket és a konzolos felhasználói felületet kezelő osztály. Itt indul a program.} % Leírás

\fajl
{Trap.java} % Kezdet
{95 byte} % Idptartam
{2015.03.21~23:00~} % Résztvevők
{A csapdák közös interface} % Leírás

\end{fajllista}

\subsection{Fordítás}

\lstset{escapeinside=`', xleftmargin=10pt, frame=single, basicstyle=\ttfamily\footnotesize, language=sh}
\begin{lstlisting}
javac -d bin *.java
\end{lstlisting}

\subsection{Futtatás}

\lstset{escapeinside=`', xleftmargin=10pt, frame=single, basicstyle=\ttfamily\footnotesize, language=sh}
\begin{lstlisting}
cd bin
java hu.stapelspeicher.main.Test
\end{lstlisting}

\section{Értékelés}


\begin{ertekeles}
\tag{Kemény} % Tag neve
{20}        % Munka szazalekban
\tag{Gema}
{20}
\tag{Pilinszki}
{20}
\tag{Somogyi}
{20}
\tag{Juszt}
{20}
\end{ertekeles}

