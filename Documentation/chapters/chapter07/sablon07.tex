% Szglab4
% ===========================================================================
%
\chapter{Prototípus koncepciója}

\thispagestyle{fancy}

\section{Prototípus interface-definíciója}
\comment{Definiálni kell a teszteket leíró nyelvet. Külön figyelmet kell fordítani arra, hogy ha a rendszer véletlen elemeket is tartalmaz, akkor a véletlenszerűség ki-bekapcsolható legyen, és a program determinisztikusan is tesztelhető legyen.}

\subsection{Az interfész általános leírása}
\comment{A protó (karakteres) input és output felületeit úgy kell kialakítani, hogy az input fájlból is vehető legyen illetőleg az output fájlba menthető legyen, vagyis kommunikációra csak a szabványos be- és kimenet használható.}

\subsection{Bemeneti nyelv}
\comment{Definiálni kell a teszteket leíró nyelvet. Külön figyelmet kell fordítani arra, hogy ha a rendszer véletlen elemeket is tartalmaz, akkor a véletlenszerűség ki-bekapcsolható legyen, és a program determinisztikusan is futtatható legyen. A szálkezelést is tesztelhető, irányítható módon kell megoldani.}

\begin{itemize}
\item Parancs1
	\begin{itemize}
	\item Leírás:
	\item Opciók:
	\end{itemize}
\item Parancs2
	\begin{itemize}
	\item Leírás:
	\item Opciók:
	\end{itemize}

\end{itemize}

\comment{Ha szükséges, meg kell adni a konfigurációs (pl. pályaképet megadó) fájlok nyelvtanát is.}

\subsection{Kimeneti nyelv}
\comment{Egyértelműen definiálni kell, hogy az egyes bemeneti parancsok végrehajtása után előálló állapot milyen formában jelenik meg a szabványos kimeneten.}

\section{Összes részletes use-case}
\comment{A use-case-eknek a részletezettsége feleljen meg a kezelői felületnek, azaz a felület elemeire kell hivatkozniuk.
Alábbi táblázat minden use-case-hez külön-külön.}

\begin{figure}[h]
\begin{center}
%\includegraphics[width=17cm]{chapters/chapter07/example.pdf}
\caption{x}
\label{fig:ProtoUseCase}
\end{center}
\end{figure}

\usecase{New Game}{Lértehoz egy pályát}{Tesztelő}{Létrehoz egy pályát a megadott méretekkel.}
\usecase{Place Cell}{Elhelyez egy cellát a pályán}{Tesztelő}{Létrehoz egy cellát és a megadott pozícióra helyezi, ha ez a pályán kívül esik #valami hiba#}
\usecase{Place Robot}{Elhelyez egy robotot a pályán}{Tesztelő}{Létrehoz egy robotot és a megadott pozícióra helyezi, ha ez a pályán kívül esik #valami hiba#. Ha a cellán már van valamilyen robot, akkor megtörténik az ütközés. Ebben az esetben a már cellán lévő foltok nem fejtik ki hatásukat.}
\usecase{Place Little Robot}{Elhelyez egy kisrobotot a pályán}{Tesztelő}{Létrehoz egy kisrobotot és a megadott pozícióra helyezi, ha ez a pályán kívül esik #valami hiba#. Ha a cellán már van valamilyen robot, akkor megtörténik az ütközés. Ebben az esetben a már cellán lévő foltok nem fejtik ki hatásukat.}
\usecase{Place Oily}{Elhelyez egy olajfoltot a pályán}{Tesztelő}{Létrehoz egy olajfoltot és a megadott pozícióra helyezi, ha ott már található folt, kicseréli azt. A cellán lévő robotokra a folt még nem fejti ki hatását.}
\usecase{Place Sticky}{Elhelyez egy ragacsfoltot a pályán}{Tesztelő}{Létrehoz egy ragacsfoltot és a megadott pozícióra helyezi, ha ott már található folt, kicseréli azt. A cellán lévő robotokra a folt még nem fejti ki hatását.}


\usecase{List Robots}{Kilistázza a pályán lévő robotokat}{Tesztelő}{A konzolon megjeleníti a pályán lévő robotok helyzetét, sebességét, a megtett távolságot és foltkészletüket.}
\usecase{List Little Robots}{Kilistázza a pályán lévő kisrobotokat}{Tesztelő}{A konzolon megjeleníti a pályán lévő kisrobotok helyzetét, sebességét, és a legközelebbi foltot ami fele tartanak.}
\usecase{List Traps}{Kilistázza a pályán lévő foltokat}{Tesztelő}{A konzolon megjeleníti a pályán lévő foltok helyzetét, típusát, és hogy mennyire kopottak vagy felszáradtak.}
\usecase{Rank List}{Kilistázza a pályán lévő robotokat rangsorát}{Tesztelő}{A konzolon megjeleníti a pályán lévő robotok által megtett utak alapján csökkenő sorrendben a robotok sorrendjét.}

\usecase{Step Robot}{Robot léptetése}{Tesztelő}{Egy megadott robot léptetése. Ha foltba lép az kifejti rá hatását, a robot pedig koptatja a ragacsfoltot. Ha másik robotra lép akkor ütköznek és egyikük kiesikHa lelép a pályáról kiesik a játékból.}
\usecase{Step Little Robot}{Kisrobot léptetése}{Tesztelő}{Egy megadott kisrobot léptetése. Ha a kisrobot elért egy foltot, addig nem lép tovább amíg fel nem takarította. Ha robotra vagy kisrobotra lép irányt vált.}



\usecase{Load Game}{Betölti a kiválasztott játékot}{Tesztelő}{A konzolon megadott játékot tölti be, ha létezik olyan. Ha nem akkor egy új játékot hoz létre.}
\usecase{Round}{Egy kör lejátszása}{Tesztelő}{Minden robot és kisrobot lép egyet, közben lezajlanak az ütközések és a foltok kifejtik hatásukat a robotokra, egyes foltok felszáradnak. A robotok ranglistája frissül a lépések alapján.}



\section{Tesztelési terv}
\comment{A tesztelési tervben definiálni kell, hogy a be- és kimeneti fájlok egybevetésével miként végezhető el a program tesztelése. Meg kell adni teszt forgatókönyveket. Az egyes teszteket elég informálisan, szabad szövegként leírni. Teszt-esetenként egy-öt mondatban. Minden teszthez meg kell adni, hogy mi a célja, a proto mely funkcionalitását, osztályait stb. teszteli. Az alábbi táblázat minden teszt-esethez külön-külön elkészítendő.}

\teszteset{...}{...}{...}

\section{Tesztelést támogató segéd- és fordítóprogramok specifikálása}
\comment{Specifikálni kell a tesztelést támogató segédprogramokat.}

