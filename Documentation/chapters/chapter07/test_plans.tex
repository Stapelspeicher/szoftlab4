
%TODO: ha a tartalma elfogadva, akkor átfogalmazni, hogy copypaste legyen
A program a standard inputra várja a parancsokat és a standard outputra teszi a kimenetét. Teszteléshez az inputot fájlból veszi és az outputot is fájlba irányítja. A tesztelő nyelvben kialakított parancsformátumok úgy lettek kialakítva, hogy teljes mértékben elősegítsék a tesztelés folyamatát. Ez alatt értendő, hogy nem csak a program egésze tesztelhető, hanem külön a részegységei is. A tesztelési forgatókönyveket a standard inputon lehet megadni szöveges formában. A helyes működést a kimenet alapján lehet eldönteni és az esetleges hibák eredetét felderíteni.

\subsection{Előforduló hibák kezelése}
A bemeneten kétféle hiba jelenhet meg:

\begin{itemize}
	
\item Szintaktikai hiba: A program futása egy hibaüzenettel leáll egy a program futása közben bekövetkezett hiba miatt.

\item Fájlbetöltési hiba: fájlkezelési probléma lép fel, végzetes hiba. A program hibaüzenettel leáll.

\end{itemize}

\subsection{Tesztelés menete}
A tesztelési forgatókönyvek a játék különböző funkcióit hivatottak tesztelni, hogy megfelelően működnek-e. Ezeket sikeresen lefuttatva a kimenetet összehasonlítva a bemenet alapján elvárt kimenettel megállapítható a helyes működés. Ha a teszteket többször lefuttatjuk, akkor megbizonyosodhatunk arról, hogy a program az adott bemenetre mindig az elvárt kimenetet eredményezi.\\
Az átfogó tesztelés végén érdemes kipróbálni a teszteket más hardvereken is, ugyan a Java programozási nyelv platformfüggetlen, de a modulok különbözősége mindig magában hordozza a hibalehetőségeket, amelyeket érdemes láthatóvá tenni majd javítani azokat. 

\teszteset{Robot hozzáadása}{Játékindítása után a robotok hozzáadódnak a pályához.}{Teszteli, hogy a robotoknak van-e kezdőpozíciójuk.}

\teszteset{Robot ragacskészlete}{A robotok a játék elején rendelkeznek bizonyos mennyiségű ragacskészlettel.}{Teszteli, hogy a robotoknak megfelelő mennyiségű ragacskészlete van-e a játék elején.}

\teszteset{Robot olajkészlete}{A robotok a játék elején rendelkeznek bizonyos mennyiségű olajkészlettel.}{Teszteli, hogy a robotoknak megfelelő mennyiségű olajkészlete van-e a játék elején.}

\teszteset{Robot megsemmisülése}{Egy robot leugrik a pályáról, megsemmisül.}{Teszteli, hogy ha egy robot leugrik a pályáról, akkor valóban megsemmisül-e.}

\teszteset{Sebességvektor módosítása (user)}{A felhasználó módosítja a robot sebességét}{Teszteli, hogy a robot sebessége valóban változott-e, és a megfelelő mértékben.}

\teszteset{Sebességvektor módosítása (ragacs)}{A ragacs felezi a sebességet.}{Teszteli, hogy a robot sebessége valóban változott-e, és valóban a felére csökkent-e.}

\teszteset{Sebességvektor módosítása (olajfolt)}{Az olajfolt nem módosítja a sebességet.}{Teszteli, hogy egy robot sebessége az olajfoltra lépés után valóban nem változtatható.}

\teszteset{Ragacs lerakása}{A robot ragacsfoltot helyez el a cellán ugrás előtt.}{Teszteli, hogy a robot valóban elhelyezi-e a ragacsfoltot, az elugrás előtt.}

\teszteset{Olajfolt lerakása}{A robot olajfoltot helyez el a cellán ugrás előtt.}{Teszteli, hogy a robot valóban elhelyezi-e az olajfoltot, az elugrás előtt.}

\teszteset{Ragacs eltűnése}{A ragacs eltűnik, miután négy robot ráugrott.}{Teszteli, hogy a ragacsfolt valóban eltűnik-e, ha pontosan négy robot ugrott már rá.}

\teszteset{Olajfolt eltűnése}{Az olajfolt bizonyos idő után felszárad.}{Teszteli, az olajfolt egy bizonyos idő letelte után valóban felszáradt-e.}

\teszteset{Kisrobotok pályára kerülése}{A pályára időnként kisrobotok ugrálnak be a foltokat feltakarítani.}{Teszteli, a robotok valóban beugrottak-e a pályára.}

\teszteset{Kisrobotok takarítása}{A kisrobotok feltakarítják a foltokat.}{Teszteli, hogy a foltok valóban eltűntek-e a pályáról, ha a kisrobot feltakarította azt.}

\teszteset{Kisrobotok indulása}{A kisrobotok a takarítás után a legközelebbi folthoz indulnak.}{Teszteli, hogy a legközelebbi folt fel veszi-e a kisrobot az irányt.}

\teszteset{Kisrobot megsemmisülése}{Ha a robot ütközik kisrobottal, akkor a kisrobot elpusztul.}{Teszteli, hogy a kisrobot megsemmisül-e.}

\teszteset{Versenyző robot versenyben maradása}{Ha a robot ütközik kisrobottal, akkor a kisrobot elpusztul, de a versenyző robot nem.}{Teszteli, hogy csak a kisrobot semmisül-e meg.}

\teszteset{Kisrobot helyén olajfolt}{Ha a robot ütközik kisrobottal, akkor a kisrobot helyén olajfolt keletkezik.}{Teszteli, hogy a kisrobot elpusztulása után keletkezett-e olajfolt.}

\teszteset{Kisrobot ütközik kisrobottal}{Ha a kisrobot másik kisrobottal ütközik, akkor irányt vált.}{Teszteli, hogy az ütköző kisrobot irányt vált-e.}

\teszteset{Robot ütközik robottal (megsemmisülés)}{Ha két robot összeütközik, akkor a lassabb megsemmisül}{Teszteli, hogy a lassabb robot megsemmisült-e.}

\teszteset{Robot ütközik robottal (sebességmódosítás)}{Ha két robot összeütközik, akkor a gyorsabb a két robot átlagsebességével folytatja a versenyt.}{Teszteli, hogy a gyorsabb robot sebessége megfelelően módosult-e.}

\teszteset{Robot ütközik kisrobottal}{A nagy robot összeütközik a kisrobottal, akkor a kicsi elpusztul}{Teszteli, hogy a kicsi valóban elpusztult-e.}








