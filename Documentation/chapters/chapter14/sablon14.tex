% Szglab4
% ===========================================================================
%
\chapter{Összefoglalás}

\thispagestyle{fancy}

\section{Projekt összegzés}

\begin{munka}
\munkaido{Gema}{99}
\munkaido{Juszt}{61}
\munkaido{Kemény}{64}
\munkaido{Pilinszki-Nagy}{64,5}
\munkaido{Somogyi}{50,5}
\osszesmunkaido{339}
\end{munka}

\begin{forrassor}
\munkaido{Szkeleton}{960}
\munkaido{Protó}{1604}
\munkaido{Grafikus}{2126}
\end{forrassor}

\begin{itemize}
\item Mit tanultak a projektből konkrétan és általában? \newline
A projekt alatt megtanultuk, hogy csapatban dolgozni nehéz dolog, és egyáltalán nem triviális feladat sem a feladatok felbontása, sem pedig azoknak a megfelelő ütemezése. Egy konkrét példa erre, amikor az elején megpróbáltunk teljesen ad-hoc “pull” stratégiával feladatokat szétosztani, ez nem ment, mondhatni csúfos kudarc lett. Méghozzá azért, mert teljesen szétesett a munkafolyamat. A félév fele körül állt be egy szabályos munkafolyamat, amivel hatékonyan tudtuk megoldani a hetente kiosztott feladatokat.
\item Mi volt a legnehezebb és a legkönnyebb? \newline
A legnehezebb része a feladatnak egyértelműen a csapatmunka volt, természetesen ez volt az egyik lényege a tárgynak. Maga a feladat egy ember számára is átlátható lett volna, így a feladatok 5 emberre való elosztása olykor hátráltatta a munkát. Mivel egy személyre túl kis egységek jutottak, nehéz volt átlátni a projektet. Ezt tovább rontották a kommunikációs nehézségek, ahol túl sok csatorna miatt nehéz volt eldönteni, hogy a legfrissebb információk hol találhatók. Így a tárggyal töltött idő nagy része sokszor felesleges kommunikációval telt. Még egy nehézség a csapatmunkával kapcsolatban az időbeosztás és a feladatok kiosztása. Mivel mindenkinek máskor volt szabadideje, nehéz volt párhuzamosítani a munkát, inkább az egymásra várakozás volt a jellemző.\newline \newline
A feladat legegyszerűbb része a kódolás maradt. 12 hét tervezgetés után a kódolás már nem jelentett problémát, azonban nem lett volna ez másképp úgy az 5. hét után sem. Többször éreztük úgy, hogy az absztrakt gondolatok helyett egyszerűbb lenne neki állni kódolni, majd az alapján tovább tervezni. Ez egy nagyobb projektnél természetesen másképp lett volna, ahol fontos az erős management és tervezés, de ez esetben a tárgyra fordítható idő erősen korlátozott volt.
\item Összhangban állt-e az idő és a pontszám az elvégzendő feladatokkal? \newline
Elméletben egy kredithez 30 munkaóra tartozik. A naplózásunk szerint egy kivételével minden csapattag túllépte az így eredményül kapott 60 munkaórát, ráadásul ebbe a konzultációkon való részvétel még nincs is beleszámolva. 
Ezt összegezve úgy éreztük, hogy a tárgy nehézsége egyáltalán nem felel meg a teljesítésével elérhető két kreditnek.\newline \newline
A naplóban visszatekintve a pontszámok sem álltak mindig teljes összhangban a befektetett munkával, példának okáért a félév elején némelyik beadandóhoz sok különféle diagrammot kellett készíteni, amelyek megtervezése és elkészítése jelentős időbefektetést igényelt, különösen azért, mert valamilyen szinten egymásra épültek.\newline \newline
Összesítve úgy véljük, hogy a tárgy alapjában vége nem egy teljesen rossz elgondolás, azonban a megvalósításán lehetne javítani, amit a változtatási javaslatok pontban részletesebben ki is fejtünk.
\item Ha nem, akkor hol okozott ez nehézséget? \newline
Az egyes fejezetekhez tartozó munka mennyisége és az erre maximálisan elérhető pontszám arányaiban megfelelt egymásnak. Viszont a tárggyal járó kreditek száma nem felelt meg a befektetett munka mennyiségének, a tárgynak legalább 3, de inkább 4 kreditesnek kellene lennie. az Ebből fakadó nehézség, hogy a beadandó fejezetek (diagramok főleg) egymásra épültek, így hiba esetén rengeteg plusz munkát okozott ezeknek a kijavítása.
\item Milyen változtatási javaslatuk van? \newline
A fenn már említett, kreditek számának a növelése.\newline \newline
A beadott programnak a végén Java6 alatt fordulnia kell, ami már kissé elavult, ma már elérhető a Java8 is, ami több lehetőséget kínál. Bár ez nem feltétlen a tárgy hibája, inkább a laborban található munkaeszközöket üzemeltetőé.\newline \newline
Amennyiben a feladat kiírása szerint a konzulens a megrendelő, akkor az a valóságtól teljesen elrugaszkodott, hogy a programunk belső struktúrájába beleszóljon, olyan módon, hogy gyakorlatilag csak az elfogadható, amit ő mond. Nyilvánvalóan konstruktív kritikának lett szánva, de ettől függetlenül a pontszámunkon ez jelentős nyomokat hagyott.

\item Milyen feladatot ajánlanának a projektre? \newline
A feladattal véleményünk szerint nem volt gond. Egy kicsit bonyolultabb feladaton ugyan jobban végig lehetne vinni a tervezési folyamatot, viszont ez az amúgy is sűrű szorgalmi időszak vége felé megbosszulná magát a program implementálásakor, így összességében a kiadott projektfeladatok megfelelő hangvételűek.
\end{itemize}

