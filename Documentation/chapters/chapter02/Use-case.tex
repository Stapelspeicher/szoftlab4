%\comment{A 2.3.1-ben felsorolt követelmények közül az alapvető és fontos követelményekhez tartozó használati esetek megadása az alábbi táblázatos formában.}
\subsection{Use-case leírások}

%\comment{Minden use-case-hez külön}

% Játékindítás
\usecase{Játékindítás}{Pályagenerálás, robotok elhelyezése, egyebek beállítása a játékkezdéshez}{Játékos(ok), Játékmotor}{A Játékos(ok) kérésere a Játékmotor új játékot indít, újra betölti/legenerálja a pályát, elhelyezi rajta a kezdeti ragacs-, és olajfoltokat és a robotokat. Egyéb tényezők is beállításra kerülnek a játékkezdéshez (idő nullázása, robotok kezdő sebességvektorának beállítása, stb.). A robotok irányítására való felhasználói felület megjelenik.}

% Sebességvektor módosítása
\usecase{Sebességvektor módosítása}{Az éppen aktív Játékos módosíthatja az saját robotjának sebességvektorát}{Aktív Játékos, Robot}{Az éppen aktív Játékos kiválaszt a felhasználói felületen egy irányt, és egy ilyen irányú irányú, egységnyi hosszú vektor fog hozzáadódni a robotjának sebességvektorához a kör végén. A sebesség nem módosítható, ha a robot olajfolton áll.}

% Ragacs hátrahagyása
\usecase{Ragacs hátrahagyása}{Az éppen aktív Játékos utasíthatja a saját robotját, hogy a következő lépés előtt hagyjon hátra ragacsfoltot.}{Aktív Játékos, Robot}{A felhasználó felület megfelelő részével az éppen aktív Játékos megadhatja, hogy a robot hagyjon-e hátra ragacsfoltot továbblépés előtt. Amennyiben a hátrahagyást választja a Játékos, a következő körben a robot elhelyez egy ragacsfoltot a jelenlegi pozícióján, majd csak ezután lép tovább.}

% Olaj hátrahagyása
\usecase{Olaj hátrahagyása}{Az éppen aktív Játékos utasíthatja a saját robotját, hogy a következő lépés előtt hagyjon hátra olajfoltot.}{Aktív Játékos, Robot}{A felhasználó felület megfelelő részével az éppen aktív Játékos megadhatja, hogy a robot hagyjon-e hátra olajfoltot továbblépés előtt.  Amennyiben a hátrahagyást választja a Játékos, a következő körben a robot elhelyez egy olajfoltot a jelenlegi pozícióján, majd csak ezután lép tovább.}

% Beállítások rögzitése
\usecase{Beállítások rögzítése}{Az éppen aktív Játékos rögzíti a robotja beállításait, innentől az adott körre vonatkozóan nem végezhet módosításokat}{Aktív Játékos, Játékmotor}{Az éppen aktív Játékos a felhasználói felület megfelelő részével rögzíti a robotjának az adott körre vonatkozó beállításait, azaz részéről lezárja a kört és átadja a beállítás lehetőségét a következő Játékosnak. Ha minden Játékos rögzítette a robotjára vonatkozó beállításokat, akkor lezárásra kerül a kör, módosításra kerülnek a sebességvektorok, majd a robotok léptetése következik.}

% Lépés
\usecase{Lépés}{A robot a sebességének megfelelő irányba lép}{Robot, Játékmotor}{A kör lezárása után a Játékmotor utasítja a robotokat a mozgásra. Ennek hatására minden robot továbblép a sebességvektor irányának megfelelő irányba, a sebességvektor nagyságával arányos távolságot. Ha előtte a robot ragacsfolton állt, akkor sebességvektora a lépés előtt megfeleződik.}

% Leugrás a pályáról
\usecase{Leugrás a pályáról}{A robot leugrik a pályáról és kiesik a játékból}{Robot, Játékmotor}{Ez a use-case a "Lépés" use-case egy kiterjesztése, akkor fordulhat elő, ha a lépés utáni pozíció a pályán kívülre mutat. Ekkor a robotnak nincs hova lépnie, tehát leugrik a pályáról, és ezáltal kiesik a játékból.}

% Játék leállítása
\usecase{Játék leállítása}{Az éppen aktív Játékos leállítja az adott játékmenetet}{Aktív Játékos, Játékmotor}{Az éppen aktív Játékos a felhasználói felület megfelelő részével leállítja az adott játékmenetet, mert már nem szeretné folytatni. A leállításról visszajelzést kap, viszont győztes robot nincs.}