\textbf{aktív robot} - az a robot, ami akciót végezhet

\textbf{akció} - a robot módosíthatja sebességét és foltot helyezhet az alatta lévő mezőre

\textbf{álló helyzet} - a robot sebessége nulla

\textbf{játék vége} - ha az összes kör lezajlott vagy egy robot kivételével az összes leugrott a pályáról a játék véget ér

\textbf{kezdőpozíció} - véletlenszerű pozíció ahol a robotok a játékot kezdik

\textbf{kör} - a játék körökre van osztva, amiben a robotok akciókat végezhetnek

\textbf{leugrás} - ha egy robot átlépi a pálya határát vagy üres mezőre lép akkor kiesik a játékból

\textbf{mező} - a pálya egységnyi területe, amin a robotok és foltok tartózkodnak

\textbf{folt} egy mezőn elhelyezkedő akadály ami a bele érkező robot mozgását befolyásolja

\textbf{olaj} - egy folt, amibe ugorva a robot nem képes megváltoztatni a sebességét

\textbf{pálya} - mezőkből álló játéktér, robotokkal és foltokkal

\textbf{ragacs} - egy folt, amibe ugorva a robot sebessége megfeleződik (alsó egész rész)

\textbf{robot} - a pályán mozgó játékos

\textbf{robot sebessége} - a robot elmozdulását meghatározó sebesség vektor

\textbf{sebesség módosítás} - a robot sebessége egységvektorral módosítható

\textbf{sebesség vektor} - az egyik mezőről a másikra mutató irányított szakasz

\textbf{ugrás} - a robot sebesség vektornyit mozdul el

\textbf{versenypálya} - véges játékmező ahol a robotok mozogni tudnak
