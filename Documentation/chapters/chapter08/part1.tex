	
\subsection{Robot hozzáadása}
\begin{itemize}
\item Leírás: \newline
Egy robot hozzáadása a pályához megadott koordinákkal. Ha a parancsban nem specifikáljuk a koordinátákat, akkor azok véletlenszerűen lesznek megadva a robotnak.
\item Ellenőrzött funkcionalitás, várható hibahelyek: \newline
Azt ellenőrizzük, hogy a robot megfelelő azonosítót és kezdőpozíciót kap-e. Hibát jelenthet az, ha olyan ID-t akarunk használni, ami már létezik, valamint ha a megadott cella már foglalt, vagy pedig nem létezik (azaz pályán kívül van)
\item Bemenet: \newline
MAP 1 1 \newline
ADDROBOT R1 5 5 0 0 \newline
ROBOTOUT R1 \newline
CELLOUT 0 0
\item Elvárt kimenet: \newline
ROBOT R1 VELOCITY=0\textunderscore0 STICKYNUM=5 OILYNUM=5 CURRCELL=0\textunderscore0 \newline
CELL 0\textunderscore0 TRAP=NULL ACTORS=R1 \newline
\end{itemize}

\subsection{Ragacs lerakása üres cellára}
\begin{itemize}
	\item Leírás: \newline
	A robot lerak egy ragacsot az aktuális cellájára.
	\item Ellenőrzött funkcionalitás, várható hibahelyek: \newline
	Ez a teszteset azt az esetet vizsgálja, amikor a robotnak van legalább egy lerakható ragacsfoltja a készletében. Hibát az jelenthet, ha valamiért nem sikerül a foltot hozzáadni az adott cellához.
	\item Bemenet: \newline
MAP 1 1 \newline
ADDROBOT R1 5 5 0 0	 \newline
ROBOTOUT R1 \newline
CELLOUT 0 0 \newline 
PLACE R1 STICKY \newline
ROBOTOUT R1 \newline
CELLOUT 0 0

	\item Elvárt kimenet: \newline
ROBOT R1 VELOCITY=0\textunderscore0 STICKYNUM=5 OILYNUM=5 CURRCELL=0\textunderscore0 \newline
CELL 0\textunderscore0 TRAP=NULL ACTORS=R1 \newline
ROBOT R1 VELOCITY=0\textunderscore0 STICKYNUM=4 OILYNUM=5 CURRCELL=0\textunderscore0 \newline
CELL 0\textunderscore0 TRAP=STICKY\textunderscore4 ACTORS=R1 

\end{itemize}

\subsection{Ragacs lerakása, már van folt a cellán
	}
\begin{itemize}
	\item Leírás: \newline
A robot lerak egy ragacsot az aktuális cellájára.
	\item Ellenőrzött funkcionalitás, várható hibahelyek: \newline
Ez a teszteset azt az esetet vizsgálja, amikor a robotnak van legalább egy lerakható ragacsfoltja a készletében. Hibát az jelenthet, ha az adott cellán már van ragacsfolt vagy olajfolt, és valamilyen oknál fogva nem váltja le az új folt a régit.

	\item Bemenet: \newline
MAP 1 1 \newline
ADDROBOT R1 5 5 0 0	 \newline
ROBOTOUT R1 \newline
ADDOILY 0 0 \newline
CELLOUT 0 0 \newline
PLACE R1 STICKY  \newline
ROBOTOUT R1 \newline
CELLOUT 0 0

	\item Elvárt kimenet: \newline
	ROBOT R1 VELOCITY=0\textunderscore0 STICKYNUM=5 OILYNUM=5 CURRCELL=0\textunderscore0 \newline
	CELL 0\textunderscore0 TRAP=OILY\textunderscore3 ACTORS=R1 \newline
	ROBOT R1 VELOCITY=0\textunderscore0 STICKYNUM=4 OILYNUM=5 CURRCELL=0\textunderscore0 \newline
	CELL 0\textunderscore0 TRAP=STICKY\textunderscore4 ACTORS=R1	
	
\end{itemize}

\subsection{Olajfolt lerakása üres cellára}
\begin{itemize}
	\item Leírás: \newline
A robot lerak egy olajfoltot az aktuális cellájára.
	\item Ellenőrzött funkcionalitás, várható hibahelyek: \newline
Ez a teszteset azt az esetet vizsgálja, amikor a robotnak van legalább egy lerakható olajfoltja a készletében. Hibát az jelenthet, ha valamiért nem sikerül a foltot hozzáadni az adott cellához.
	\item Bemenet: \newline
MAP 1 1 \newline
ADDROBOT R1 5 5 0 0	 \newline
ROBOTOUT R1 \newline
CELLOUT 0 0 \newline
PLACE R1 OILY \newline
ROBOTOUT R1 \newline
CELLOUT 0 0
	\item Elvárt kimenet: \newline
ROBOT R1 VELOCITY=0\textunderscore0 STICKYNUM=5 OILYNUM=5 CURRCELL=0\textunderscore0 \newline
CELL 0\textunderscore0 TRAP=NULL ACTORS=R1 \newline
ROBOT R1 VELOCITY=0\textunderscore0 STICKYNUM=5 OILYNUM=4 CURRCELL=0\textunderscore0 \newline
CELL 0\textunderscore0 TRAP=OILY\textunderscore3 ACTORS=R1

\end{itemize}


\subsection{Olaj lerakása, már van folt a cellán}
\begin{itemize}
	\item Leírás: \newline
A robot lerak egy olajfoltot az aktuális cellájára.

	\item Ellenőrzött funkcionalitás, várható hibahelyek: \newline
Ez a teszteset azt az esetet vizsgálja, amikor a robotnak van legalább egy lerakható olajfoltja a készletében. Hibát az jelenthet, ha az adott cellán már van ragacsfolt vagy olajfolt, és valamilyen oknál fogva nem váltja le az új folt a régit.

	\item Bemenet: \newline
MAP 1 1 \newline
ADDROBOT R1 5 5 0 0	 \newline
ROBOTOUT R1 \newline
ADDSTICKY 0 0 \newline
CELLOUT 0 0 \newline
PLACE R1 OILY \newline
CELLOUT R1

	\item Elvárt kimenet: \newline
ROBOT R1 VELOCITY=0\textunderscore0 STICKYNUM=5 OILYNUM=5 CURRCELL=0\textunderscore0 \newline
CELL 0\textunderscore0 TRAP=STICKY\textunderscore4 ACTORS=R1 \newline 
ROBOT R1 VELOCITY=0\textunderscore0 STICKYNUM=5 OILYNUM=4 CURRCELL=0\textunderscore0 \newline
CELL 0\textunderscore0 TRAP=OILY\textunderscore3 ACTORS=R1	

\end{itemize}

\subsection{Robot megsemmisülése}
\begin{itemize}
	\item Leírás: \newline
A robot leugrik a pályáról, ezzel megsemmisül
	\item Ellenőrzött funkcionalitás, várható hibahelyek: \newline
Azt ellenőrizzük, hogy a pálya elhagyása valóban a robot megsemmisülését eredményezi-e. Problémát jelenthet, ha a robot nem regisztrálódik ki a celláról valamilyen oknál fogva.

	\item Bemenet: \newline
MAP 1 1 \newline
ADDROBOT R1 0 0 0 0 \newline
ROBOTOUT R1 \newline
STEP R1 -X \newline
ROBOTOUT R1 \newline
CELLOUT 0 0

	\item Elvárt kimenet: \newline
ROBOT R1 VELOCITY=0\textunderscore0 STICKYNUM=5 OILYNUM=5 CURRCELL=0\textunderscore0 \newline
ROBOT R1 VELOCITY=-1\textunderscore0 STICKYNUM=5 OILYNUM=4 CURRCELL=NULL \newline
CELL 0\textunderscore0 TRAP=OILY\textunderscore3 ACTORS=NULL \newline

\end{itemize}

\subsection{Sebességvektor módosítása (user)}
\begin{itemize}
	\item Leírás: \newline
 A user sebességet változtat és lép a robottal.

	\item Ellenőrzött funkcionalitás, várható hibahelyek: \newline
Ellenőrizzük, hogy helyesen működik-e a program azon funkciója, melyet használva a user módosíthatja a robot sebességvektorát. Az jelenthet potenciális hibát, ha a robot éppen olajfolton áll.

	\item Bemenet: \newline
MAP 2 1  \newline
ADDROBOT R1 0 0 0 0 \newline
CELLOUT 0 0 \newline
ROBOTOUT R1 \newline 
STEP R1 X \newline
ROBOTOUT R1 \newline
CELLOUT 0 0 \newline
CELLOUT 1 0

	
	\item Elvárt kimenet: \newline
CELL 0\textunderscore0 TRAP=NULL ACTORS=R1 \newline
ROBOT R1 VELOCITY=0\textunderscore0 STICKYNUM=0 OILYNUM=0 CURRCELL=0\textunderscore0 \newline
ROBOT R1 VELOCITY=1\textunderscore0 STICKYNUM=0 OILYNUM=0 CURRCELL=1\textunderscore0 \newline
CELL 0\textunderscore0 TRAP=NULL ACTORS=NULL \newline
CELL 1\textunderscore0 TRAP=NULL ACTORS=R1 \newline

	
\end{itemize}


