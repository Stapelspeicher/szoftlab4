	
\subsection{Robot hozzáadása}
\begin{itemize}
\item Leírás: \newline
Egy robot hozzáadása a pályához megadott koordinákkal. Ha a parancsban nem specifikáljuk a koordinátákat, akkor azok véletlenszerűen lesznek megadva a robotnak.
\item Ellenőrzött funkcionalitás, várható hibahelyek: \newline
Azt ellenőrizzük, hogy a robot megfelelő azonosítót és kezdőpozíciót kap-e. Hibát jelenthet az, ha olyan ID-t akarunk használni, ami már létezik, valamint ha a megadott cella már foglalt, vagy pedig nem létezik (azaz pályán kívül van)
\item Bemenet: \newline
MAP 1 1 \newline
ADDROBOT R1 5 5 0 0 \newline
ROBOTOUT R1 \newline
CELLOUT 0 0
\item Elvárt kimenet: \newline
ROBOT R1 VELOCITY=0\textunderscore0 STICKYNUM=5 OILYNUM=5 CURRCELL=0\textunderscore0 \newline
CELL 0\textunderscore0 TRAP=NULL ACTORS=R1 \newline
\end{itemize}

\subsection{Ragacs lerakása üres cellára}
\begin{itemize}
	\item Leírás: \newline
	A robot lerak egy ragacsot az aktuális cellájára.
	\item Ellenőrzött funkcionalitás, várható hibahelyek: \newline
	Ez a teszteset azt az esetet vizsgálja, amikor a robotnak van legalább egy lerakható ragacsfoltja a készletében. Hibát az jelenthet, ha valamiért nem sikerül a foltot hozzáadni az adott cellához.
	\item Bemenet: \newline
MAP 1 1 \newline
ADDROBOT R1 5 5 0 0	 \newline
ROBOTOUT R1 \newline
CELLOUT 0 0 \newline 
PLACE R1 STICKY \newline
ROBOTOUT R1 \newline
CELLOUT 0 0

	\item Elvárt kimenet: \newline
ROBOT R1 VELOCITY=0\textunderscore0 STICKYNUM=5 OILYNUM=5 CURRCELL=0\textunderscore0 \newline
CELL 0\textunderscore0 TRAP=NULL ACTORS=R1 \newline
ROBOT R1 VELOCITY=0\textunderscore0 STICKYNUM=4 OILYNUM=5 CURRCELL=0\textunderscore0 \newline
CELL 0\textunderscore0 TRAP=STICKY\textunderscore4 ACTORS=R1 

\end{itemize}

\subsection{Ragacs lerakása, már van folt a cellán
	}
\begin{itemize}
	\item Leírás: \newline
A robot lerak egy ragacsot az aktuális cellájára.
	\item Ellenőrzött funkcionalitás, várható hibahelyek: \newline
Ez a teszteset azt az esetet vizsgálja, amikor a robotnak van legalább egy lerakható ragacsfoltja a készletében. Hibát az jelenthet, ha az adott cellán már van ragacsfolt vagy olajfolt, és valamilyen oknál fogva nem váltja le az új folt a régit.

	\item Bemenet: \newline
MAP 1 1 \newline
ADDROBOT R1 5 5 0 0	 \newline
ROBOTOUT R1 \newline
ADDOILY 0 0 \newline
CELLOUT 0 0 \newline
PLACE R1 STICKY  \newline
ROBOTOUT R1 \newline
CELLOUT 0 0

	\item Elvárt kimenet: \newline
	ROBOT R1 VELOCITY=0\textunderscore0 STICKYNUM=5 OILYNUM=5 CURRCELL=0\textunderscore0 \newline
	CELL 0\textunderscore0 TRAP=OILY\textunderscore3 ACTORS=R1 \newline
	ROBOT R1 VELOCITY=0\textunderscore0 STICKYNUM=4 OILYNUM=5 CURRCELL=0\textunderscore0 \newline
	CELL 0\textunderscore0 TRAP=STICKY\textunderscore4 ACTORS=R1	
	
\end{itemize}

\subsection{Olajfolt lerakása üres cellára}
\begin{itemize}
	\item Leírás: \newline
A robot lerak egy olajfoltot az aktuális cellájára.
	\item Ellenőrzött funkcionalitás, várható hibahelyek: \newline
Ez a teszteset azt az esetet vizsgálja, amikor a robotnak van legalább egy lerakható olajfoltja a készletében. Hibát az jelenthet, ha valamiért nem sikerül a foltot hozzáadni az adott cellához.
	\item Bemenet: \newline
MAP 1 1 \newline
ADDROBOT R1 5 5 0 0	 \newline
ROBOTOUT R1 \newline
CELLOUT 0 0 \newline
PLACE R1 OILY \newline
ROBOTOUT R1 \newline
CELLOUT 0 0
	\item Elvárt kimenet: \newline
ROBOT R1 VELOCITY=0\textunderscore0 STICKYNUM=5 OILYNUM=5 CURRCELL=0\textunderscore0 \newline
CELL 0\textunderscore0 TRAP=NULL ACTORS=R1 \newline
ROBOT R1 VELOCITY=0\textunderscore0 STICKYNUM=5 OILYNUM=4 CURRCELL=0\textunderscore0 \newline
CELL 0\textunderscore0 TRAP=OILY\textunderscore3 ACTORS=R1

\end{itemize}


\subsection{Olaj lerakása, már van folt a cellán}
\begin{itemize}
	\item Leírás: \newline
A robot lerak egy olajfoltot az aktuális cellájára.

	\item Ellenőrzött funkcionalitás, várható hibahelyek: \newline
Ez a teszteset azt az esetet vizsgálja, amikor a robotnak van legalább egy lerakható olajfoltja a készletében. Hibát az jelenthet, ha az adott cellán már van ragacsfolt vagy olajfolt, és valamilyen oknál fogva nem váltja le az új folt a régit.

	\item Bemenet: \newline
MAP 1 1 \newline
ADDROBOT R1 5 5 0 0	 \newline
ROBOTOUT R1 \newline
ADDSTICKY 0 0 \newline
CELLOUT 0 0 \newline
PLACE R1 OILY \newline
ROBOTOUT R1 \newline
CELLOUT 0 0

	\item Elvárt kimenet: \newline
ROBOT R1 VELOCITY=0\textunderscore0 STICKYNUM=5 OILYNUM=5 CURRCELL=0\textunderscore0 \newline
CELL 0\textunderscore0 TRAP=STICKY\textunderscore4 ACTORS=R1 \newline 
ROBOT R1 VELOCITY=0\textunderscore0 STICKYNUM=5 OILYNUM=4 CURRCELL=0\textunderscore0 \newline
CELL 0\textunderscore0 TRAP=OILY\textunderscore3 ACTORS=R1	

\end{itemize}

\subsection{Robot megsemmisülése}
\begin{itemize}
	\item Leírás: \newline
A robot leugrik a pályáról, ezzel megsemmisül
	\item Ellenőrzött funkcionalitás, várható hibahelyek: \newline
Azt ellenőrizzük, hogy a pálya elhagyása valóban a robot megsemmisülését eredményezi-e. Problémát jelenthet, ha a robot nem regisztrálódik ki a celláról valamilyen oknál fogva.

	\item Bemenet: \newline
MAP 1 1 \newline
ADDROBOT R1 0 0 0 0 \newline
ROBOTOUT R1 \newline
STEP R1 -X \newline
ROBOTOUT R1 \newline
CELLOUT 0 0

	\item Elvárt kimenet: \newline
ROBOT R1 VELOCITY=0\textunderscore0 STICKYNUM=0 OILYNUM=0 CURRCELL=0\textunderscore0 \newline
ROBOT R1 VELOCITY=-1\textunderscore0 STICKYNUM=0 OILYNUM=0 CURRCELL=0\textunderscore0 \newline
CELL 0\textunderscore0 TRAP=NULL ACTORS=NULL \newline

\end{itemize}

\subsection{Sebességvektor módosítása (user)}
\begin{itemize}
	\item Leírás: \newline
 A user sebességet változtat és lép a robottal.

	\item Ellenőrzött funkcionalitás, várható hibahelyek: \newline
Ellenőrizzük, hogy helyesen működik-e a program azon funkciója, melyet használva a user módosíthatja a robot sebességvektorát. Az jelenthet potenciális hibát, ha a robot éppen olajfolton áll.

	\item Bemenet: \newline
MAP 2 1  \newline
ADDROBOT R1 0 0 0 0 \newline
CELLOUT 0 0 \newline
ROBOTOUT R1 \newline 
STEP R1 X \newline
ROBOTOUT R1 \newline
CELLOUT 0 0 \newline
CELLOUT 1 0

	
	\item Elvárt kimenet: \newline
CELL 0\textunderscore0 TRAP=NULL ACTORS=R1 \newline
ROBOT R1 VELOCITY=0\textunderscore0 STICKYNUM=0 OILYNUM=0 CURRCELL=0\textunderscore0 \newline
ROBOT R1 VELOCITY=1\textunderscore0 STICKYNUM=0 OILYNUM=0 CURRCELL=1\textunderscore0 \newline
CELL 0\textunderscore0 TRAP=NULL ACTORS=NULL \newline
CELL 1\textunderscore0 TRAP=NULL ACTORS=R1 \newline

	
\end{itemize}

\subsection{Sebességvektor módosítása (ragacs)}
\begin{itemize}
	\item Leírás: \newline
	A ragacsfolt megfelezi a robot aktuális sebességvektorát.
	\item Ellenőrzött funkcionalitás, várható hibahelyek: \newline 
	Azt a funkcionalitást ellenőrizzük, miszerint a ragacsfolt megfelezi a robot jelenlegi sebességvektorát. Problémát jelenthet az, ha páratlan az egyik koordináta. Meg kell vizsgálni, hogy ilyenkor jól kerekíti-e lefele a program.
	
	\item Bemenet: \newline
		MAP 4 1 \newline
		ADDSTICKY 3 0 \newline
		ADDROBOT R1 0 0 0 0 \newline
		ROBOTOUT R1  \newline
		STEP R1 X \newline
		ROBOTOUT R1 \newline
		STEP R1 X \newline
		ROBOTOUT R1 \newline
		
	\item Elvárt kimenet: \newline
	ROBOT R1 VELOCITY=0\textunderscore0 STICKYNUM=0 OILYNUM=0 CURRCELL=0\textunderscore0 \newline
	ROBOT R1 VELOCITY=1\textunderscore0 STICKYNUM=0 OILYNUM=0 CURRCELL=1\textunderscore0 \newline
	ROBOT R1 VELOCITY=1\textunderscore0 STICKYNUM=0 OILYNUM=0 CURRCELL=3\textunderscore0 \newline
	
\end{itemize}

\subsection{Sebességvektor módosítása (olajfolt)}
\begin{itemize}
	\item Leírás: \newline
	A robot olajfoltra lép, és ezért olajos lesz, tehát a sebességét a következő lépésnél nem lehet majd módosítani.
	\item Ellenőrzött funkcionalitás, várható hibahelyek: \newline 
	Ez a teszt azt ellenőrzni, hogy valóban megakadályozza-e a sebesség változtatását az, ha a robot éppen egy olajfolton tartózkodik. Várható hiba, hogy esetleg mégis sikerül a usernek ezt megváltoztatnia.
	\item Bemenet: \newline
		MAP 4 1 \newline
		ADDOILY 1 0 \newline
		ADDROBOT R1 0 0 0 0 \newline
		STEP R1 X \newline
		ROBOTOUT R1 \newline
		STEP R1 X \newline
		ROBOTOUT R1 \newline
	\item Elvárt kimenet: \newline
	
		ROBOT R1 VELOCITY=1\textunderscore0 STICKYNUM=0 OILYNUM=0 CURRCELL=1\textunderscore0 \newline
		ROBOT R1 VELOCITY=1\textunderscore0 STICKYNUM=0 OILYNUM=0 CURRCELL=2\textunderscore0 \newline
		
\end{itemize}

\subsection{Ragacs lerakása(nincs elég ragacs)}
\begin{itemize}
	\item Leírás: \newline
	 A robot megpróbál lerakni egy ragacsfoltot, azonban nincs a készletében egy se.
	\item Ellenőrzött funkcionalitás, várható hibahelyek: \newline 
	Azt az esetet teszteljük, amikor 0 ragacskészlettel rendelkező robot mégiscsak megpróbálkozik ragacsfolt elhelyezésével, ami nyilvánvalóan nem sikerülhet neki.
	
	\item Bemenet: \newline
	MAP 2 2 \newline
	ADDROBOT R1 0 0 0 0 \newline
	ROBOTOUT R1 \newline
	PLACE R1 STICKY \newline
	CELLOUT 0 0 \newline
 	ROBOTOUT R1 \newline
	
	\item Elvárt kimenet: \newline
	ROBOT R1 VELOCITY=0\textunderscore0 STICKYNUM=0 OILYNUM=0 CURRCELL=0\textunderscore0 \newline
	CELL 0\textunderscore0 TRAP=NULL ACTORS=R1 \newline
	ROBOT R1 VELOCITY=0\textunderscore0 STICKYNUM=0 OILYNUM=0 CURRCELL=0\textunderscore0 \newline
\end{itemize}

\subsection{Olaj lerakása (nincs elég olaj)}
\begin{itemize}
	\item Leírás: \newline
	 A robot megpróbál lerakni egy olajfoltot, azonban nincs a készletében egy se.
	\item Ellenőrzött funkcionalitás, várható hibahelyek: \newline 
	Azt az esetet teszteljük, amikor 0 olajkészlettel rendelkező robot mégiscsak megpróbálkozik olajfolt elhelyezésével, ami nyilvánvalóan nem 
	\item Bemenet: \newline
	MAP 2 2 \newline
	ADDROBOT R1 0 0 0 0 \newline
	ROBOTOUT R1 \newline
	PLACE R1 OILY \newline
	CELLOUT 0 0 \newline
	ROBOTOUT R1 \newline
	\item Elvárt kimenet: \newline
	ROBOT R1 VELOCITY=0\textunderscore0 STICKYNUM=0 OILYNUM=0 CURRCELL=0\textunderscore0 \newline
	CELL 0\textunderscore0 TRAP=NULL ACTORS=R1 \newline
	ROBOT R1 VELOCITY=0\textunderscore0 STICKYNUM=0 OILYNUM=0 CURRCELL=0\textunderscore0 \newline
\end{itemize}

\subsection{Ragacs eltűnése (elkopik)}
\begin{itemize}
	\item Leírás: \newline
	A ragacsfolt mindig egyre jobban elkopik, amikor egy robot rálép. Amikor eléri a nullás szintet, akkor teljesen eltűnik.
	\item Ellenőrzött funkcionalitás, várható hibahelyek: \newline 
	Azt teszteljük, hogy a ragacsfolt valóban elkopik-e. Ezt úgy tehetjük meg, hogy robotokat léptetünk rá, minden léptetés után lekérdezve a cella állapotát.
	\item Bemenet: \newline
	MAP 2 2
		ADDSTICKY 0 0 \newline
		CELLOUT 0 0 \newline
		ADDROBOT R1 0 0 1 0 \newline
		STEP R1 -X \newline
		CELLOUT 0 0 \newline
		ADDROBOT R2 0 0 1 0 \newline
		STEP R1 -X \newline
		STEP R2 -X \newline
		CELLOUT 0 0 \newline
		ADDROBOT R3 0 0 1 0 \newline
		STEP R2 -X \newline
		STEP R3 -X \newline
		CELLOUT 0 0 \newline
		ADDROBOT R4 0 0 1 0 \newline
		STEP R3 -X \newline
		STEP R4 -X \newline
		CELLOUT 0 0 \newline
	\item Elvárt kimenet: \newline
	CELL 0\textunderscore0 TRAP=STICKY\textunderscore4 ACTORS=NULL
	\newline
	CELL 0\textunderscore0 TRAP=STICKY\textunderscore3 ACTORS=R1
	\newline
	CELL 0\textunderscore0 TRAP=STICKY\textunderscore2 ACTORS=R2
	\newline
	CELL 0\textunderscore0 TRAP=STICKY\textunderscore1 ACTORS=R3
	\newline
	CELL 0\textunderscore0 TRAP=NULL ACTORS=R4
	
\end{itemize}

\subsection{Olajfolt eltűnése (felszárad)}
\begin{itemize}
	\item Leírás: \newline
	Az olajfolt körönként egyre jobban felszárad. Amikor eléri a nullás szintet, akkor teljesen eltűnik. 
	\item Ellenőrzött funkcionalitás, várható hibahelyek: \newline 
	 Azt teszteljük, hogy az olajfolt valóban felszárad-e. Ezt a körök léptetésével tehetjük meg, minden léptetés után lekérdezve a cella állapotát.
	 Esetleges hibát jelenthet az, ha ezalatt az idő alatt egy másik robot rálép a cellára és elhelyez egy másik csapdát.
	\item Bemenet: \newline
	MAP 2 2 \newline
	ADDROBOT R1 0 1 0 0 \newline
	ROBOTOUT R1 \newline
	PLACE R1 OILY \newline
	ROBOTOUT R1 \newline
	CELLOUT 0 0 \newline
	ENDTURN \newline
	CELLOUT 0 0 \newline
	ENDTURN \newline
	CELLOUT 0 0 \newline
	ENDTURN \newline
	CELLOUT 0 0 \newline
	ENDTURN \newline
	CELLOUT 0 0 \newline
	\item Elvárt kimenet: \newline
	ROBOT R1 VELOCITY=0\textunderscore0 STICKYNUM=0 OILYNUM=1 CURRCELL=0\textunderscore0
	\newline
	ROBOT R1 VELOCITY=0\textunderscore0 STICKYNUM=0 OILYNUM=0 
	CURRCELL=0\textunderscore0
	\newline
	CELL 0\textunderscore0 TRAP=OILY\textunderscore4 ACTORS=R1
	\newline
	CELL 0\textunderscore0 TRAP=OILY\textunderscore3 ACTORS=R1
	\newline
	CELL 0\textunderscore0 TRAP=OILY\textunderscore2 ACTORS=R1
	\newline
	CELL 0\textunderscore0 TRAP=OILY\textunderscore1 ACTORS=R1
	\newline
	CELL 0\textunderscore0 TRAP=NULL ACTORS=R1
\end{itemize}
