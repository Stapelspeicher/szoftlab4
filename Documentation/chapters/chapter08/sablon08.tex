% Szglab4
% ===========================================================================
%
\chapter{Részletes tervek}

\thispagestyle{fancy}

\section{Osztályok és metódusok tervei}

\subsection{GameMap}
\begin{itemize}
\item Felelősség\newline
Az osztály nyilvántartja a pályát (a cellákat), valamint tárolja a körök számát.

\item Attribútumok
	\begin{itemize}
		\item - cells: Cell[][]: A pálya celláit tároló mátrix.
		\item - rounds: int: A hátralevő körök száma.
		\item - xlength: int: A pálya X irányú kiterjedése.
		\item - ylength: int: A pálya Y irányú kiterjedése.
	\end{itemize}
\item Metódusok
	\begin{itemize}
		\item + GameMap(Integer x, Integer y):  A GameMap két paraméteres konstruktora, beállítja a körök számát egy előre definiált értékre.
		\item + setRounds(int rounds): void: Beállít egy bizonyos mennyiségű körszámot a játék elején.
		\item + getCell(Position p): Cell: Visszadja a pálya egy adott celláját, vagy nullt, ha az adott helyen nincsen cella.
		\item + addCell(Position p): void: Cella hozzáadása a pályához a paraméterként átadott pozícióra.
		\item + getFreeNeighbouringCell(Position p): Cell: Az adott pozíció körül visszad egy üres cellát, vagy null-t, ha nincs ilyen.
		\item + decrementRounds(): boolean: Körök számának csökkentése. Ha még nem fogytak el a körök, akkor visszatér true-val, különben false. 
	\end{itemize}
\end{itemize}

\subsection{Cell}
\begin{itemize}
	\item Felelősség\newline
	A pályát felépítő elemek megvalósítása, és a rálépő objektumokkal való interakció a felelőssége.

	\item Attribútumok
		
	\begin{itemize}
		\item - actors: List<ActiveObject>: A cellán álló mozgó objektumokat tároló lista.
		\item - trap: Trap: A cellára elhelyezett csapda (ha van).
		\item - pos: Position: A cella pozíciója a pályán.
		\item - map: GameMap: A GameMap (pálya) osztály azon objektuma, amelyben a cella taálható.
	\end{itemize}
	\item Metódusok

	\begin{itemize}
		\item + setPosition(Position pos): void: A cella pozícióját állítja be.
		\item + setMap(GameMap gm): void: A cellához tatozó GameMap-et állítja be, a játék GameMap paraméter alapján.
		\item + add(ActiveObject ao): void: A cellán egy újabb mozgó objektumot helyez el, ha cellán folt található, akkor az kifejti hatását, valamint a mozgó elemeket ütközteti egymással. Paraméterként a cellához hozzáadni kívánt objektumot adjuk meg.
		\item + add(Trap t): void: A cellán egy új foltot helyez el, olyan fajtát, amilyet paraméterként átadunk.
		\item + getCellFromHere(Position p): Cell: A cellától megadott vektor távolságra lévő cellát adja meg, amit a GameMap-től kér el. 
		\item + remove(ActiveObject ao): void: A celláról eltávolítja a paraméterként átadott mozgó objektumot.
		\item + isEmpty(): boolean: Teszteli, hogy a cellán vannak-e mozgó objektumok. Ha a cella üres, akkor visszatér true-val. 
		\item + getDistanceFromCell(Cell c): double: Két cella közötti távolságot adja meg, úgy, hogy a két koordinátát egymásból kivonja, majd a különbségeket összegzi. Ha az eredmény negatív lenne, akkor egy -1-szeres szorzással ennek könnyen az abszolút értékét vehetjük.
		\item + getFreeNeighbouringCell(): Cell: Egy szabad szomszédos cellát ad vissza, amit a GameMap.től kér el. Ha ilyen nem létezik, akkor null-al tér vissza.
		\item + removeTrap(): void: Csapda törlése.
		
	\end{itemize}
\end{itemize}


\subsection{Robot}
\begin{itemize}
	\item Felelősség\newline
	A Robot osztály felelőssége egy robot reprezentációja, és a robot tevékenységeinek biztosítása kifele, azaz interakció a felhasználóval, a csapdákkal, és ütközés esetén a többi mozgó objektummal.
	
	\item Interfészek\newline
	ActiveObject
	\item Attribútumok
	\begin{itemize}
		\item - velocity: Position: A robot vektoriális sebessége.
		\item - currCell: Cell: A cella, amin a robot aktuálisan áll.
		\item - oily: boolean: Megadja, hogy a robot olajfolton áll-e.
		\item - stickyNum: int: A robot rendelkezésére álló ragacsfoltok száma.
		\item - oilNum: int: A robot rendelkezésére álló olajfoltok száma.
		\item - distance: int: A robot által megtett távolság.
		
	\end{itemize}
	\item Metódusok\newline
	\begin{itemize}
		\item + Robot(int stickyNum, int oilyNum): A Robot osztály kétparaméteres konstruktora, paraméterként megadható, hogy a robot hány ragacs- és olajfolttal induljon a verseny elején.
		\item + placeSticky(): void: Ha a robotnak van ragacsa, akkor azt elhelyezi azon a cellán, amelyikről épp elugrani készül.
		\item + placeOily(): void: Ha a robotnak van olajfoltja, akkor azt elhelyezi azon a cellán, amelyikről épp elugrani készül.
		\item + addVelocity(Position p): void: Ha a robot nem olajfolton áll, akkor megváltozik a sebessége a paraméterként átadott irányban és nagyságban.
		\item + stepOn(ActiveObject ao): void: Kezeli esetet, amikor két mozgó objektum ütközik egymással.
		\item + oilyEffect(): void: A robot, amelyik egy olajfoltos cellára léphet, az az adott körben nem változtathatja a sebességét. 
		\item + stickyEffect(): void: Hatására a a robot sebessége feleződik. 
		\item + collideWithRobot(Robot other): void: Kezeli az esetet, amikor egy robot egy másik mozgó objektummal ütközik. Ha nincsen szabad szomszédos cella, akkor mind a két robot megsemmisül. 
		\item + die(): void: Az objektum megsemmisül: egy robot leesik a pályáról, vagy olyan módon ütköznek egymással, hogy mind a két robot megsemmisül. 
		\item + setCell(Cell c): void: Beállítja a mozgó objektum celláját, azaz, hogy éppen melyik cellán tartózkodik.
		\item + step(): void: A lépést ténylegesen lebonyolítja. Ha a robot lelép a pályáról, akkor megsemmisül. Különben a megtett távolsága növekszik és megváltozik az aktuális cellája, amin éppen áll.
		\item + getCell(): Cell: Megadja, hogy éppen melyik cellán áll a mozgó objektum.
		\item + getDistance(): int: A robot aktuális sebességéről tájékoztat. 
		\item + collideWithLittleRobot(LittleRobot other): void: Kisrobottal ütköztet.
		
	\end{itemize}
\end{itemize}

\subsection{Oily}
\begin{itemize}
	\item Felelősség\newline
	Olajfolt osztály, felelőssége a rálépő robottal való interakció (sebességmódosítás megtiltása).
	\item Interfészek\newline
	Trap  - Csapda interface, amelyik összefogja az egyes akadályozó elemeket és biztosítja az egységes kezelésüket.
	\item Attribútumok
	\begin{itemize}
		\item - wetness: int: Méri az olajfolt tartósságat, azaz, hogy még hány körön képes kiváltani az olajfolt a hatását.
	\end{itemize}
	\item Metódusok
	\begin{itemize}
		\item + stepOn(ActiveObject ao): void: Biztosítja az olajfolt specifikus viselkedését: a rálépő robot sebessége nem módosítható.
		\item + dry(): boolean: Az olajfolt tartósságát csökkenti, ha ez egyenlő nullával, akkor visszatér.
		item + abrade(): boolean: Koptatja az olajfoltot.
	\end{itemize}
\end{itemize}


\subsection{Sticky}
\begin{itemize}
	\item Felelősség
	
	Ragacs osztály, felelőssége a rálépő robottal való interakció (a sebesség felezése).
	\item Interfészek
	
	Trap - Csapda interface, amelyik összefogja az egyes akadályozó elemeket és biztosítja az egységes kezelésüket.
	
	\item Attribútumok
	\begin{itemize}
		\item - abrasion: int: A ragacs tartóssága.
	\end{itemize}

	\item Metódusok
	\begin{itemize}
		\item + stepOn(ActiveObject ao): void: Biztosítja a ragacsfolt specifikus viselkedését: megfelezi a rálépő robot sebességet.
		\item + dry(): boolean: Visszatér false-al, a ragacs nem szárad.
		\item: + abrade(): boolean: Koptatja a ragacsot, ha tartóssága elérte a nullát, akkor visszatér.
	\end{itemize}
\end{itemize}

\subsection{Position}
\begin{itemize}
	\item Felelősség
	Koordinátapárokat reprezentáló osztály, amely ezekkel műveletek végez. (immutable)
	\item Attribútumok
	\begin{itemize}
		\item -x: int: X koordináta.
		\item -y: int: Y koordináta.
	\end{itemize}
	\item Metódusok
	\begin{itemize}
		\item + Position(int x, int y): a Position osztály konstruktora.
		\item + getX(): int: megadja az x koordinátát.
		\item + getY(): int: megadja az y koordinátát.
		\item + divide(Position p): Position: A koordináták osztását végző metódus, az osztó paraméterként adható meg.
		\item + add(Position p): Position: A paraméterként átadott értékkel növeli a koordinátákat.
		\item + getDistance(Position p): double: A paraméterként átadott ponthoz mért távolság nagyságát adja meg.
		\item + compareTo(Position other): int: Két Position objektumot hasonlít össze a vektor hossza alapján. Ha az aktuális objektum hossza nagyobb, mint a paraméterként kapott, akkor 1 a visszatérési érték, ha egyenlőek, akkor 0, különben -1.
	\end{itemize}
\end{itemize}

\subsection{Logger}
\begin{itemize}
	\item Felelősség\newline

	\item Attribútumok\newline
	
	\begin{itemize}
		\item \underline{- depth: int}
		\item \underline{+ objectNames: Map<String, String>}
	\end{itemize}
	\item Metódusok
	\begin{itemize}
		\item \underline{- addSpaces(): void}
		\item \underline{+ enterFunction(String s, Object o): void}
		\item \underline{+ exitFunction(): void}
	\end{itemize}
\end{itemize}

\subsection{Test}
\begin{itemize}
	\item Felelősség
		A tesztelésért felelős osztály.
	
	\item Attribútumok
	\begin{itemize}
		\item \underline{- in: Scanner}
	\end{itemize}
	\item Metódusok
	\begin{itemize}
		\item \underline{+ main(String[] args): void:} Ebben a függvényben indul a program, kiírja a főmenüt, meghívja a teszteket, vagy a kért almenüt. A parancssori argumentumokat nem használja.
		\item \underline{- collision(): void:} Két robot ütközését szimulálja. Megkérdezi a felhasználót, hogy legyen-e az ugráshoz üres szomszédos cella, majd eszerint felépíti a pályát.
		\item \underline{- step(): void:} Kiírja a robotléptetés menüjét és meghívja a kívánt tesztelő függvényt.
		\item \underline{- stepsOnEmptyCell(): void:} Azt az esetet szimulálja, amikor a robot leugrik a pályáról.
		\item \underline{- stepsOnOilyCell(): void:} A robot egy olajfoltos cellára lép, ütközés nem történik.
		\item \underline{- stepsOnStickyCell(): void:} A robot egy ragacsfoltos cellára lép, ütközés nem történik.
		\item \underline{- stepsOnPlainCell() :void:} A robot egy folt nélküli cellára lép és ütközés nem történik.
		\item\underline{ - nextRound(): void:} Teszteli a Következő kör use-caset, akkor is, ha van még hátra kör, és akkor is, ha már nincs, ezt a felhasználó válasza dönti el.
		\item \underline{- placeSticky(): void:} A robot megpróbál letenni egy cellára egy ragacsfoltot. A felhasználó döntése alapján ekkor vagy rendelkezik ragaccsal, vagy nem.
		\item \underline{- placeOily(): void:}: A robot megpróbál letenni egy cellára egy olajfoltot. A felhasználó döntése alapján ekkor vagy rendelkezik olajjal, vagy nem.
		\item \underline{- changeRobotSpeed(): void:} Egy robot sebességet változtat.
		\item \underline{- getIndex(int bound): void:} Beolvassa a felhasználó által beírt menüpontot, s megcizsgálja, hogy az valós-e. Paraméterként a legnagyobb választható menüpont sorszáma adható át.
		\item \underline{- launchGame(): void:} A játékindítás menüpont almenüjének kiírása és a választott menüponthoz tartozó függvény hívása.
		\item  \underline{- createRobot(): void:} Létrehoz egy robotot a a felhasználó által megadott ragacs- és olajkészlettel, és elhelyezi a pályára.
		\item \underline{- createCell(): void:} Létrehoz egy cellát a felhasználó által megadott koordinátákra. 
		\item \underline{- printIndex(): int:} A főmenüt kiíró függvény. Bekéri a menüpont sorszámát és visszaadja az őt hívó függvénynek.
		
	\end{itemize}
\end{itemize}

\subsection{LittleRobot}
\begin{itemize}
	\item Felelősség
	A kisrobot osztály.
	
	\item Attribútumok
	\begin{itemize}
		\item - LittleRobotState: state: A kisrobot állapota.
		\item - dazedCounter: int: Körök száma, ameddig a kisrobot kábult állapotban van.
		\item - cleaningCounter: int: A körök száma, ameddig a takarítás tart.
		\item - currCell: Cell: A kisrobot éppen aktuális cellája.
	\end{itemize}
	\item Metódusok
	\begin{itemize}
		\item - startCleaningIfNeeded(): void: A kisrobot elkezd takarítani, ha az szükséges. A kisrobot állapota CLEANING lesz.
		\item + setState(LittleRobotState newState): void: A kisrobot állapotát állítja át.
		\item + stepOn(ActiveObject ao): void: Az ütközést kezeli le.
		\item + oilyEffect(): void: Kiváltódik az olajfolt hatása, a kisrobotra ez semmi.
		\item + stickyEffect(): void: Kiváltódik a ragacsfolt hatása, ez a kisrobot esetén semmi.
		\item + collideWithRobot(Robot other): void: Ütközés hatására a kisrobot kábult állapotba kerül, ha van szabad cella, akkor odakerül, különben megsemmmisül.
		\item + die(): void: A kisrobot megsemmisül.
		\item + setCell(Cell c): void: A kisrobot blép a pályára, a legközelebbi csapdához igyekszik mindig.
		\item + step(): void: A kisrobot lép. Különböző állapotaiban (normál, cleaning, dazed) másképpen teszi ezt.
		\item + getCell(): Cell: Visszadja azt a cellát, amelyiken a kisrobot éppen áll.
		\item + collideWithLittleRobot(LittleRobot other): void: Kisrobot ütközik kisrobottal, ha van üres szomszédos cella, akkor odakerül és kábult állapotba lép, ha nincs ilyen, akkor megsemmisül.
		
	\end{itemize}
\end{itemize}


\section{A tesztek részletes tervei, leírásuk a teszt nyelvén}
[A tesztek részletes tervei alatt meg kell adni azokat a bemeneti adatsorozatokat, amelyekkel a program működése ellenőrizhető. Minden bemenő adatsorozathoz definiálni kell, hogy az adatsorozat végrehajtásától a program mely részeinek, funkcióinak ellenőrzését várjuk és konkrétan milyen eredményekre számítunk, ezek az eredmények hogyan vethetők össze a bemenetekkel.]

	
\subsection{Robot hozzáadása}
\begin{itemize}
\item Leírás: \newline
Egy robot hozzáadása a pályához megadott koordinákkal. Ha a parancsban nem specifikáljuk a koordinátákat, akkor azok véletlenszerűen lesznek megadva a robotnak.
\item Ellenőrzött funkcionalitás, várható hibahelyek: \newline
Azt ellenőrizzük, hogy a robot megfelelő azonosítót és kezdőpozíciót kap-e. Hibát jelenthet az, ha olyan ID-t akarunk használni, ami már létezik, valamint ha a megadott cella már foglalt, vagy pedig nem létezik (azaz pályán kívül van)
\item Bemenet: \newline
MAP 1 1 \newline
ADDROBOT R1 5 5 0 0 \newline
ROBOTOUT R1 \newline
CELLOUT 0 0
\item Elvárt kimenet: \newline
ROBOT R1 VELOCITY=0\textunderscore0 STICKYNUM=5 OILYNUM=5 CURRCELL=0\textunderscore0 \newline
CELL 0\textunderscore0 TRAP=NULL ACTORS=R1 \newline
\end{itemize}

\subsection{Ragacs lerakása üres cellára}
\begin{itemize}
	\item Leírás: \newline
	A robot lerak egy ragacsot az aktuális cellájára.
	\item Ellenőrzött funkcionalitás, várható hibahelyek: \newline
	Ez a teszteset azt az esetet vizsgálja, amikor a robotnak van legalább egy lerakható ragacsfoltja a készletében. Hibát az jelenthet, ha valamiért nem sikerül a foltot hozzáadni az adott cellához.
	\item Bemenet: \newline
MAP 1 1 \newline
ADDROBOT R1 5 5 0 0	 \newline
ROBOTOUT R1 \newline
CELLOUT 0 0 \newline 
PLACE R1 STICKY \newline
ROBOTOUT R1 \newline
CELLOUT 0 0

	\item Elvárt kimenet: \newline
ROBOT R1 VELOCITY=0\textunderscore0 STICKYNUM=5 OILYNUM=5 CURRCELL=0\textunderscore0 \newline
CELL 0\textunderscore0 TRAP=NULL ACTORS=R1 \newline
ROBOT R1 VELOCITY=0\textunderscore0 STICKYNUM=4 OILYNUM=5 CURRCELL=0\textunderscore0 \newline
CELL 0\textunderscore0 TRAP=STICKY\textunderscore4 ACTORS=R1 

\end{itemize}

\subsection{Ragacs lerakása, már van folt a cellán
	}
\begin{itemize}
	\item Leírás: \newline
A robot lerak egy ragacsot az aktuális cellájára.
	\item Ellenőrzött funkcionalitás, várható hibahelyek: \newline
Ez a teszteset azt az esetet vizsgálja, amikor a robotnak van legalább egy lerakható ragacsfoltja a készletében. Hibát az jelenthet, ha az adott cellán már van ragacsfolt vagy olajfolt, és valamilyen oknál fogva nem váltja le az új folt a régit.

	\item Bemenet: \newline
MAP 1 1 \newline
ADDROBOT R1 5 5 0 0	 \newline
ROBOTOUT R1 \newline
ADDOILY 0 0 \newline
CELLOUT 0 0 \newline
PLACE R1 STICKY  \newline
ROBOTOUT R1 \newline
CELLOUT 0 0

	\item Elvárt kimenet: \newline
	ROBOT R1 VELOCITY=0\textunderscore0 STICKYNUM=5 OILYNUM=5 CURRCELL=0\textunderscore0 \newline
	CELL 0\textunderscore0 TRAP=OILY\textunderscore3 ACTORS=R1 \newline
	ROBOT R1 VELOCITY=0\textunderscore0 STICKYNUM=4 OILYNUM=5 CURRCELL=0\textunderscore0 \newline
	CELL 0\textunderscore0 TRAP=STICKY\textunderscore4 ACTORS=R1	
	
\end{itemize}

\subsection{Olajfolt lerakása üres cellára}
\begin{itemize}
	\item Leírás: \newline
A robot lerak egy olajfoltot az aktuális cellájára.
	\item Ellenőrzött funkcionalitás, várható hibahelyek: \newline
Ez a teszteset azt az esetet vizsgálja, amikor a robotnak van legalább egy lerakható olajfoltja a készletében. Hibát az jelenthet, ha valamiért nem sikerül a foltot hozzáadni az adott cellához.
	\item Bemenet: \newline
MAP 1 1 \newline
ADDROBOT R1 5 5 0 0	 \newline
ROBOTOUT R1 \newline
CELLOUT 0 0 \newline
PLACE R1 OILY \newline
ROBOTOUT R1 \newline
CELLOUT 0 0
	\item Elvárt kimenet: \newline
ROBOT R1 VELOCITY=0\textunderscore0 STICKYNUM=5 OILYNUM=5 CURRCELL=0\textunderscore0 \newline
CELL 0\textunderscore0 TRAP=NULL ACTORS=R1 \newline
ROBOT R1 VELOCITY=0\textunderscore0 STICKYNUM=5 OILYNUM=4 CURRCELL=0\textunderscore0 \newline
CELL 0\textunderscore0 TRAP=OILY\textunderscore3 ACTORS=R1

\end{itemize}


\subsection{Olaj lerakása, már van folt a cellán}
\begin{itemize}
	\item Leírás: \newline
A robot lerak egy olajfoltot az aktuális cellájára.

	\item Ellenőrzött funkcionalitás, várható hibahelyek: \newline
Ez a teszteset azt az esetet vizsgálja, amikor a robotnak van legalább egy lerakható olajfoltja a készletében. Hibát az jelenthet, ha az adott cellán már van ragacsfolt vagy olajfolt, és valamilyen oknál fogva nem váltja le az új folt a régit.

	\item Bemenet: \newline
MAP 1 1 \newline
ADDROBOT R1 5 5 0 0	 \newline
ROBOTOUT R1 \newline
ADDSTICKY 0 0 \newline
CELLOUT 0 0 \newline
PLACE R1 OILY \newline
CELLOUT R1

	\item Elvárt kimenet: \newline
ROBOT R1 VELOCITY=0\textunderscore0 STICKYNUM=5 OILYNUM=5 CURRCELL=0\textunderscore0 \newline
CELL 0\textunderscore0 TRAP=STICKY\textunderscore4 ACTORS=R1 \newline 
ROBOT R1 VELOCITY=0\textunderscore0 STICKYNUM=5 OILYNUM=4 CURRCELL=0\textunderscore0 \newline
CELL 0\textunderscore0 TRAP=OILY\textunderscore3 ACTORS=R1	

\end{itemize}

\subsection{Robot megsemmisülése}
\begin{itemize}
	\item Leírás: \newline
A robot leugrik a pályáról, ezzel megsemmisül
	\item Ellenőrzött funkcionalitás, várható hibahelyek: \newline
Azt ellenőrizzük, hogy a pálya elhagyása valóban a robot megsemmisülését eredményezi-e. Problémát jelenthet, ha a robot nem regisztrálódik ki a celláról valamilyen oknál fogva.

	\item Bemenet: \newline
MAP 1 1 \newline
ADDROBOT R1 0 0 0 0 \newline
ROBOTOUT R1 \newline
STEP R1 -X \newline
ROBOTOUT R1 \newline
CELLOUT 0 0

	\item Elvárt kimenet: \newline
ROBOT R1 VELOCITY=0\textunderscore0 STICKYNUM=5 OILYNUM=5 CURRCELL=0\textunderscore0 \newline
ROBOT R1 VELOCITY=-1\textunderscore0 STICKYNUM=5 OILYNUM=4 CURRCELL=NULL \newline
CELL 0\textunderscore0 TRAP=OILY\textunderscore3 ACTORS=NULL \newline

\end{itemize}

\subsection{Sebességvektor módosítása (user)}
\begin{itemize}
	\item Leírás: \newline
 A user sebességet változtat és lép a robottal.

	\item Ellenőrzött funkcionalitás, várható hibahelyek: \newline
Ellenőrizzük, hogy helyesen működik-e a program azon funkciója, melyet használva a user módosíthatja a robot sebességvektorát. Az jelenthet potenciális hibát, ha a robot éppen olajfolton áll.

	\item Bemenet: \newline
MAP 2 1  \newline
ADDROBOT R1 0 0 0 0 \newline
CELLOUT 0 0 \newline
ROBOTOUT R1 \newline 
STEP R1 X \newline
ROBOTOUT R1 \newline
CELLOUT 0 0 \newline
CELLOUT 1 0

	
	\item Elvárt kimenet: \newline
CELL 0\textunderscore0 TRAP=NULL ACTORS=R1 \newline
ROBOT R1 VELOCITY=0\textunderscore0 STICKYNUM=0 OILYNUM=0 CURRCELL=0\textunderscore0 \newline
ROBOT R1 VELOCITY=1\textunderscore0 STICKYNUM=0 OILYNUM=0 CURRCELL=1\textunderscore0 \newline
CELL 0\textunderscore0 TRAP=NULL ACTORS=NULL \newline
CELL 1\textunderscore0 TRAP=NULL ACTORS=R1 \newline

	
\end{itemize}

\subsection{Sebességvektor módosítása (ragacs)}
\begin{itemize}
	\item Leírás: \newline
	A ragacsfolt megfelezi a robot aktuális sebességvektorát.
	\item Ellenőrzött funkcionalitás, várható hibahelyek: \newline 
	Azt a funkcionalitást ellenőrizzük, miszerint a ragacsfolt megfelezi a robot jelenlegi sebességvektorát. Problémát jelenthet az, ha páratlan az egyik koordináta. Meg kell vizsgálni, hogy ilyenkor jól kerekíti-e lefele a program.
	
	\item Bemenet: \newline
		MAP 2 1 \newline
		ADDROBOT R1 0 0 0 0 \newline
		ADDSTICKY 1 0 \newline
		STEP R1 X \newline
		ROBOTOUT R1 \newline
		
	\item Elvárt kimenet: \newline
	ROBOT R1 VELOCITY=0\textunderscore0 STICKYNUM=0 OILYNUM=0 CURRCELL=0\textunderscore0 \newline
	
\end{itemize}

\subsection{Sebességvektor módosítása (olajfolt)}
\begin{itemize}
	\item Leírás: \newline
	A robot olajfoltra lép, és ezért olajos lesz, tehát a sebességét a következő lépésnél nem lehet majd módosítani.
	\item Ellenőrzött funkcionalitás, várható hibahelyek: \newline 
	Ez a teszt azt ellenőrzni, hogy valóban megakadályozza-e a sebesség változtatását az, ha a robot éppen egy olajfolton tartózkodik. Várható hiba, hogy esetleg mégis sikerül a usernek ezt megváltoztatnia.
	\item Bemenet: \newline
		MAP 3 1 \newline
		ADDROBOT 0 0 0 0 \newline
		ADDOILY 1 0 \newline
		ROBOT STEP X \newline
		ROBOTOUT R1 \newline
		ROBOT STEP X \newline
		ROBOTOUT R1 \newline
	\item Elvárt kimenet: \newline
	
		ROBOT R1 VELOCITY=1\textunderscore0 STICKYNUM=0 OILYNUM=0 CURRCELL=1\textunderscore0 \newline
		ROBOT R1 VELOCITY=1\textunderscore0 STICKYNUM=0 OILYNUM=0 CURRCELL=2\textunderscore0 \newline
		
\end{itemize}

\subsection{Ragacs lerakása(nincs elég ragacs)}
\begin{itemize}
	\item Leírás: \newline
	 A robot megpróbál lerakni egy ragacsfoltot, azonban nincs a készletében egy se.
	\item Ellenőrzött funkcionalitás, várható hibahelyek: \newline 
	Azt az esetet teszteljük, amikor 0 ragacskészlettel rendelkező robot mégiscsak megpróbálkozik ragacsfolt elhelyezésével, ami nyilvánvalóan nem sikerülhet neki.
	
	\item Bemenet: \newline
	MAP 2 2 \newline
	ADDROBOT R1 0 0 0 0 \newline
	ROBOTOUT R1 \newline
	PLACE R1 STICKY \newline
	CELLOUT 0 0 \newline
 	ROBOTOUT R1 \newline
	
	\item Elvárt kimenet: \newline
	ROBOT R1 VELOCITY=0\textunderscore0 STICKYNUM=0 OILYNUM=0 CURRCELL=0\textunderscore0 \newline
	CELL 0\textunderscore0 TRAP=NULL ACTORS=R1 \newline
	ROBOT R1 VELOCITY=0\textunderscore0 STICKYNUM=0 OILYNUM=0 CURRCELL=0\textunderscore0 \newline
\end{itemize}

\subsection{Olaj lerakása (nincs elég olaj)}
\begin{itemize}
	\item Leírás: \newline
	 A robot megpróbál lerakni egy olajfoltot, azonban nincs a készletében egy se.
	\item Ellenőrzött funkcionalitás, várható hibahelyek: \newline 
	Azt az esetet teszteljük, amikor 0 olajkészlettel rendelkező robot mégiscsak megpróbálkozik olajfolt elhelyezésével, ami nyilvánvalóan nem 
	\item Bemenet: \newline
	MAP 2 2 \newline
	ADDROBOT R1 0 0 0 0 \newline
	ROBOTOUT R1 \newline
	PLACE R1 OILY \newline
	CELLOUT 0 0 \newline
	ROBOTOUT R1 \newline
	\item Elvárt kimenet: \newline
	ROBOT R1 VELOCITY=0\textunderscore0 STICKYNUM=0 OILYNUM=0 CURRCELL=0\textunderscore0 \newline
	CELL 0\textunderscore0 TRAP=NULL ACTORS=R1 \newline
	ROBOT R1 VELOCITY=0\textunderscore0 STICKYNUM=0 OILYNUM=0 CURRCELL=0\textunderscore0 \newline
\end{itemize}

\subsection{Ragacs eltűnése (elkopik)}
\begin{itemize}
	\item Leírás: \newline
	A ragacsfolt mindig egyre jobban elkopik, amikor egy robot rálép. Amikor eléri a nullás szintet, akkor teljesen eltűnik.
	\item Ellenőrzött funkcionalitás, várható hibahelyek: \newline 
	Azt teszteljük, hogy a ragacsfolt valóban elkopik-e. Ezt úgy tehetjük meg, hogy robotokat léptetünk rá, minden léptetés után lekérdezve a cella állapotát.
	\item Bemenet: \newline
	MAP 2 2  \newline
	ADDROBOT R1 1 0 0 0 \newline
	ADDSTICKY 0 0 \newline
	CELLOUT 0 0 \newline
	STEP R1 -X \newline
	CELLOUT 0 0 \newline
	ADDROBOT R2 1 0 0 0 \newline
	STEP R1 -X \newline
	STEP R2 -X \newline
	CELLOUT 0 0 \newline
	ADDROBOT R3 1 0 0 0 \newline
	STEP R2 -X \newline
	STEP R3 -X \newline
	CELLOUT 0 0 \newline
	ADDROBOT R4 1 0 0 0 \newline
	STEP R3 -X \newline
	STEP R4 -X \newline
	CELLOUT 0 0 \newline
	\item Elvárt kimenet: \newline
	CELL 0\textunderscore0 TRAP=STICKY\textunderscore4 ACTORS=NULL
	\newline
	CELL 0\textunderscore0 TRAP=STICKY\textunderscore3 ACTORS=R1
	\newline
	CELL 0\textunderscore0 TRAP=STICKY\textunderscore2 ACTORS=R2
	\newline
	CELL 0\textunderscore0 TRAP=STICKY\textunderscore1 ACTORS=R3
	\newline
	CELL 0\textunderscore0 TRAP=NULL ACTORS=R4
	
\end{itemize}

\subsection{Olajfolt eltűnése (felszárad)}
\begin{itemize}
	\item Leírás: \newline
	Az olajfolt körönként egyre jobban felszárad. Amikor eléri a nullás szintet, akkor teljesen eltűnik. 
	\item Ellenőrzött funkcionalitás, várható hibahelyek: \newline 
	 Azt teszteljük, hogy az olajfolt valóban felszárad-e. Ezt a körök léptetésével tehetjük meg, minden léptetés után lekérdezve a cella állapotát.
	 Esetleges hibát jelenthet az, ha ezalatt az idő alatt egy másik robot rálép a cellára és elhelyez egy másik csapdát.
	\item Bemenet: \newline
	MAP 2 2 \newline
	ADDROBOT R1 0 1 0 0 \newline
	ROBOTOUT R1 \newline
	PLACE R1 OILY \newline
	ROBOTOUT R1 \newline
	CELLOUT 0 0 \newline
	ENDTURN \newline
	CELLOUT 0 0 \newline
	ENDTURN \newline
	CELLOUT 0 0 \newline
	ENDTURN \newline
	CELLOUT 0 0 \newline
	ENDTURN \newline
	CELLOUT 0 0 \newline
	\item Elvárt kimenet: \newline
	ROBOT R1 VELOCITY=0\textunderscore0 STICKYNUM=0 OILYNUM=1 CURRCELL=0\textunderscore0
	\newline
	ROBOT R1 VELOCITY=0\textunderscore0 STICKYNUM=0 OILYNUM=0 
	CURRCELL=0\textunderscore0
	CELL 0\textunderscore0 TRAP=OILY\textunderscore4 ACTORS=R1
	\newline
	CELL 0\textunderscore0 TRAP=OILY\textunderscore3 ACTORS=R1
	\newline
	CELL 0\textunderscore0 TRAP=OILY\textunderscore2 ACTORS=R1
	\newline
	CELL 0\textunderscore0 TRAP=OILY\textunderscore1 ACTORS=R1
	\newline
	CELL 0\textunderscore0 TRAP=NULL ACTORS=R1
\end{itemize}


\subsection{Teszteset1}
\begin{itemize}
\item Leírás\newline
\comment{szöveges leírás, kb. 1-5 mondat.}
\item Ellenőrzött funkcionalitás, várható hibahelyek
\item Bemenet\newline
\comment{a proto bemeneti nyelvén megadva (lásd előző anyag)}
\item Elvárt kimenet\newline
\comment{a proto kimeneti nyelvén megadva (lásd előző anyag)}
\end{itemize}

\subsection{Teszteset2}
\begin{itemize}
\item Leírás\newline
\comment{szöveges leírás, kb. 1-5 mondat.}
\item Ellenőrzött funkcionalitás, várható hibahelyek
\item Bemenet\newline
\comment{a proto bemeneti nyelvén megadva (lásd előző anyag)}
\item Elvárt kimenet\newline
\comment{a proto kimeneti nyelvén megadva (lásd előző anyag)}
\end{itemize}

\section{A tesztelést támogató programok tervei}
\comment{A tesztadatok előállítására, a tesztek eredményeinek kiértékelésére szolgáló segédprogramok részletes terveit kell elkészíteni.}

