% Szglab4
% ===========================================================================
%
\chapter{Prototípus beadása}

\thispagestyle{fancy}

\section{Módosítások a tervhez képest}

A debug mód (azaz a DBG\_ON és DBG\_OFF, mint parancsori paraméterek) kikerültek a programból, mivel szükségtelennek bizonyultak. Helyette az első argumentum értéke lehet "single" vagy "multiple". Ezzel biztosítjuk, hogy lehessen a program egyszer indításával több tesztet is futtatni, és ne legyen szükség egy külön segédprogramra.\par
Single módban az single kulcsszó utáni első paraméter értéke a specifikált nyelven megírt tesztesetet tartalmazó fájl, a második pedig az a fájl, aminek a teszt eredményét kell tartalmaznia.\par
A multiple módon belül két külön módot különböztethetünk meg. Ebben a módban a multiple kulcsszó utáni első paraméter a specifikált nyelven megírt teszteseteket tartalmazó mappa, a második pedig az a mappa, amibe az eredményeknek kell kerülniük. Az teszteseket kiterjesztésének mindenképpen \textbf{.txt}-nek kell lennie. A kimeneti fájlok a multiple kulcsszó utáni második paraméterként megadott mappába fognak kerülni. Nevük ugyanaz lesz, mint a hozzájuk tartozó tesztesetnek, kiterjesztésük pedig \textbf{.result.txt} lesz.\par
A multiple másik módjában a program generál egy HTML fájlt a tesztek eredményeiről, ami a kimeneti mappába kerül \textbf{results.html} néven. Ezt a módot úgy lehet aktiválni, hogy a multiple kulcsó utáni harmadik paraméterként megadunk egy mappát, amelyik mindegyik tesztesethez tartalmazza az elvárt kimenetet pontosan ugyanabban a formátumban, mint amit a program kimenetként kiad. (Tehát a teszteset fájljával megegyező név és \textbf{.result.txt} kiterjesztés)\par
\par
A leadott fájlok között a \textbf{test/testcases} mappában megtalálhatók a tesztesetek, a \textbf{test/expected} mappában pedig az elvárt kimenetek.

\section{Fordítási és futtatási útmutató}

\subsection{Fájllista}

\begin{fajllista}

\fajl
{ActiveObject.java} % Kezdet
{1588 byte} % Idptartam
{2015.03.21~23:00~} % Résztvevők
{A játék mozgó résztvevőinek a közös interface} % Leírás

\fajl
{Cell.java} % Kezdet
{5025 byte} % Idptartam
{2015.03.21~23:00~} % Résztvevők
{Az cellákat reprezentáló osztály} % Leírás

\fajl
{GameMap.java} % Kezdet
{3438 byte} % Idptartam
{2015.03.21~23:00~} % Résztvevők
{A játékteret reprezentáló osztály} % Leírás

\fajl
{LittleRobot.java}
{3313 byte}
{2015.04.10~20:07~}
{A kisrobotokat megvalósító osztály}

\fajl
{LittleRobotState.java}
{96 byte}
{2015.04.10~20:07~}
{A kisrobotok állapotait megvalósító enumeráció}

\fajl
{Logger.java} % Kezdet
{1023 byte} % Idptartam
{2015.03.21~23:00~} % Résztvevők
{A függvényhívások konzolra Loggolásához használt osztály} % Leírás

\fajl
{Oily.java} % Kezdet
{761 byte} % Idptartam
{2015.03.21~23:00~} % Résztvevők
{Az olajfoltot reprezentáló osztály} % Leírás

\fajl
{Position.java} % Kezdet
{2661 byte} % Idptartam
{2015.03.21~23:00~} % Résztvevők
{A játéktér koordinátáit reprezentáló osztály} % Leírás

\fajl
{ProtoController.java}
{9708 byte}
{2015.04.10~20:07~}
{A megírt teszteket végrehajtó és kiértékelő osztály. Ez a program belépési pontja.}

\fajl
{Robot.java} % Kezdet
{4836 byte} % Idptartam
{2015.03.21~23:00~} % Résztvevők
{A Robotokat megvalósító osztály} % Leírás

\fajl
{Sticky.java} % Kezdet
{809 byte} % Idptartam
{2015.03.21~23:00~} % Résztvevők
{A Ragacs csapdát megvalósító osztály} % Leírás

\fajl
{Test.java} % Kezdet
{10544 byte} % Idptartam
{2015.03.21~23:00~} % Résztvevők
{A teszteket és a konzolos felhasználói felületet kezelő osztály. A szkeleton teszteléséhez használt.} % Leírás

\fajl
{Trap.java} % Kezdet
{706 byte} % Idptartam
{2015.03.21~23:00~} % Résztvevők
{A csapdák közös interface} % Leírás

\end{fajllista}

\subsection{Fordítás}

\lstset{escapeinside=`', xleftmargin=10pt, frame=single, basicstyle=\ttfamily\footnotesize, language=sh}
\begin{lstlisting}
javac -d bin *.java
\end{lstlisting}

\subsection{Futtatás}

{Egy adott tesztesethez:
\lstset{escapeinside=`', xleftmargin=10pt, frame=single, basicstyle=\ttfamily\footnotesize, language=sh}
\begin{lstlisting}
cd bin
java hu.stapelspeicher.main.ProtoController single `[tesztfájl] [kimeneti fájl]
\end{lstlisting}

Több tesztesethez (ellenőrzés nélkül):
\lstset{escapeinside=`', xleftmargin=10pt, frame=single, basicstyle=\ttfamily\footnotesize, language=sh}
\begin{lstlisting}
cd bin
java hu.stapelspeicher.main.ProtoController multiple `[tesztkönyvtár] [kimeneti könyvtár]
\end{lstlisting}

Több tesztesethez (ellenőrzéssel):
\lstset{escapeinside=`', xleftmargin=10pt, frame=single, basicstyle=\ttfamily\footnotesize, language=sh}
\begin{lstlisting}
cd bin
java hu.stapelspeicher.main.ProtoController multiple `[tesztkönyvtár] [kimeneti könyvtár] [elvárt eredmények könyvtára]
\end{lstlisting}


\section{Tesztek jegyzőkönyvei}
\subsection{Robot hozzáadása}
\tesztok
{Gema}
{2015.05.19~15:10}

\subsection{Ragacs lerakása üres cellára}
\tesztok
{Gema}
{2015.05.19~15:10}

\subsection{Ragacs lerakása, már van folt a cellán}
\tesztok
{Gema}
{2015.05.19~15:10}

\subsection{Olajfolt lerakása üres cellára}
\tesztok
{Gema}
{2015.05.19~15:10}

\subsection{Olaj lerakása, már van folt a cellán}
\tesztok
{Gema}
{2015.05.19~15:10}

\subsection{Robot megsemmisülése}
\tesztok
{Gema}
{2015.05.19~15:10}

\subsection{Sebességvektor módosítása (user)}
\tesztok
{Gema}
{2015.05.19~15:10}

\subsection{Sebességvektor módosítása (ragacs)}
\tesztok
{Gema}
{2015.05.19~15:10}

\subsection{Sebességvektor módosítása (olajfolt)}
\tesztok
{Gema}
{2015.05.19~15:10}

\subsection{Ragacs lerakása (nincs elég ragacs)}
\tesztok
{Gema}
{2015.05.19~15:10}

\subsection{Olaj lerakása (nincs elég olaj)}
\tesztok
{Gema}
{2015.05.19~15:10}

\subsection{Ragacs eltűnése (elkopik)}
\tesztok
{Gema}
{2015.05.19~15:10}

\subsection{Olajfolt eltűnése (felszárad)}
\tesztok
{Gema}
{2015.05.19~15:10}

\subsection{Olajfolt eltűnése (felszárad)}
\tesztok
{Gema}
{2015.05.19~15:10}

\subsection{Kisrobotok takarítása}
\tesztok
{Gema}
{2015.05.19~15:10}

\subsection{Kisrobotok indulása}
\tesztok
{Gema}
{2015.05.19~15:10}

\tesztfail
{Gema}
{2015.04.19.~13:32}
{NullPointerException a LittleRobot.java 87. sorában}
{A getNearestTrapRelativePostion() null-lal is visszatérhet, ami nem lett lekezelve a setCell() függvényben}
{null visszatérés lekezelése}

\tesztfail
{Gema}
{2015.04.19.~13:44}
{NullPointerException a LittleRobot.java 117. sorában}
{A getNearestTrapRelativePostion() null-lal is visszatérhet, ami nem lett lekezelve a step() függvényben}
{null visszatérés lekezelése}

\subsection{Kisrobot megsemmisül}
\tesztok
{Gema}
{2015.05.19~15:10}

\tesztfail
{Gema}
{2015.04.19.~15:47}
{Elvárt kimenet:\newline
CELL 1\_0 TRAP=OILY\_4 ACTORS=R1\newline
CELL 0\_0 TRAP=NULL ACTORS=NULL\newline
Valódi kimenet:\newline
CELL 1\_0 TRAP=NULL ACTORS=R1\newline
CELL 0\_0 TRAP=OILY\_4 ACTORS=NULL\newline
Hiba: Rossz cellára került az olajfolt a kisrobot halálakor.
}
{A hiba oka az előző esethez hasonlóan az, hogy a Robot osztály step() metódusában newCell.add(this) és a currCell=newCell sorok rossz sorrendben szerepelnek.}
{A két említett sor megcserélése.}

\subsection{Robot ütközik kisrobottal}
\tesztok
{Gema}
{2015.05.19~15:10}

\subsection{Kisrobot ütközik kisrobottal}
\tesztok
{Gema}
{2015.05.19~15:10}

\tesztfail
{Gema}
{2015.04.19.~13:47}
{NullPointerException a LittleRobot.java 123. sorában}
{A Cell osztály getNearestTrapRelativePosition metódusa relatív helyett abszolút pozícióval tért vissza, így előfordulhatott, hogy a pozíció egy nem létező cellára mutatott.}
{Abszolút pozíció javítása relatívra, valamint a Position osztályba egy subtract(Position p) metódus implementálása}

\subsection{Robot ütközik robottal}
\tesztok
{Gema}
{2015.05.19~15:10}

\subsection{Kisrobotra takarítás közben kisrobot lép}
\tesztok
{Gema}
{2015.04.19.~15:10}

\tesztfail
{Gema}
{2015.04.19.~15:10}
{Elvárt kimenet:\newline
LITTLEROBOT LR1 STATE=CLEANING DAZEDCOUNTER=0 CLEANINGCOUNTER=2 CURRCELL=1\_1\newline
LITTLEROBOT LR1 STATE=CLEANING DAZEDCOUNTER=0 CLEANINGCOUNTER=1 CURRCELL=1\_1\newline
LITTLEROBOT LR2 STATE=DAZED DAZEDCOUNTER=2 CLEANINGCOUNTER=0 CURRCELL=0\_1\newline
Valódi kimenet:\newline
LITTLEROBOT LR1 STATE=CLEANING DAZEDCOUNTER=0 CLEANINGCOUNTER=2 CURRCELL=1\_1\newline
LITTLEROBOT LR1 STATE=CLEANING DAZEDCOUNTER=0 CLEANINGCOUNTER=1 CURRCELL=1\_1\newline
LITTLEROBOT LR2 STATE=CLEANING DAZEDCOUNTER=2 CLEANINGCOUNTER=2 CURRCELL=1\_1\newline
Hiba: Nem frissült a kisrobot helyzete az utolsó sorban az ütközés után.}
{A hiba oka, hogy a LittleRobot osztály step() metódusában a c.add(this) függvényhívás megelőzte a currCell=c értékadást. Ez azért jelentett problémát, mert az add() metódus hívja meg az ütköztető metódusokat, amik felhasználják a kisrobot jelenlegi helyzetét. Mivel az értékadás csak a függvényhívás után következett, ezért rossz pozíciót használt az ütköztető függvény.}
{A két említett sor megcserélése. Megjegyzés: Ezzel helyessé váltak az egyébként még nem megvizsgált 'Kisrobot ütközik kisrobottal' és a 'Robot ütközik kisrobottal' tesztesetek is.}

\subsection{Kisrobot takarít és egy robot ráugrik közben}
\tesztok
{Gema}
{2015.04.19.~15:10}

\tesztfail
{Gema}
{2015.04.19.~13:40}
{NullPointerException a LittleRobot.java 41. sorában}
{A getNearestTrapRelativePostion() null-lal is visszatérhet, ami nem lett lekezelve a startCleaningIfNeeded() függvényben}
{null visszatérés lekezelése}

\section{Értékelés}
\comment{A projekt kezdete óta az értékelésig eltelt időben tagokra bontva, százalékban.}

\begin{ertekeles}
\tag{Gema} % Tag neve
{25}        % Munka szazalekban
\tag{Kemény}
{20}
\tag{Juszt}
{20}
\tag{Pilinszki-Nagy}
{18}
\tag{Somogyi}
{17}
\end{ertekeles}

