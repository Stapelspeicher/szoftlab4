A szkeleton egyes funkcióit egy menüből lehet majd elérni a parancssoron keresztül. A program indítása után megjelenik a főmenü a 8 menüponttal, amik közül a menüpont számának megadásával (és egy enter lenyomásával) választhat a felhasználó. A menü a következőképpen fog kinézni:\\

1. Játékindítás\\
\indent \hspace{1 cm}1.1 Cella hozzáadása\\
\indent \hspace{2 cm} 1.1.1 Cella koordinátái szóközzel elválasztva:\\
\indent \hspace{1 cm} 1.2 Robot hozzáadása\\
\indent \hspace{1 cm} 1.3 Főmenübe lépés\\
\indent 2. Robot sebességének módosítása\\
\indent \hspace{1 cm} 2.1 A sebesség koordinátái szóközzel elválasztva:\\
\indent 3. Olajfolt elhelyezése\\
\indent \hspace{1 cm} 3.1 Van olajfoltja a robotnak? $(I/N)$\\
\indent 4. Ragacsfolt elhelyezése\\
\indent \hspace{1 cm} 4.1 Van ragacsfoltja a robotnak? $(I/N$)\\
\indent 5. Következő kör\\
\indent \hspace{1 cm} 5.1 Ez volt az utolsó kör? $(I/N)$\\
\indent 6. Lépés\\
\indent \hspace{1 cm} 6.1 Üres sima cellára lép a robot\\
\indent \hspace{1 cm} 6.2 Ragacsos cellára lép a robot\\
\indent \hspace{1 cm} 6.3 Olajos cellára lép a robot\\
\indent \hspace{1 cm} 6.4 Pályán kívülre lép a robot\\
\indent 7. Ütközés\\
\indent \hspace{1 cm}	7.1 Van elég szabad hely a szomszéd cellákban? $(I/N)$\\	
\indent 8. Kilépés a játékból\\

Az almenüvel rendelkező menüpontok esetén a menüpont kiválasztása után megjelenik az almenü, aminek a kezelése a menüpontoktól függően változik. Abban az esetben, ha egy eldöntendő kérdést ír ki a program, akkor az $I$ billentyűvel adhat igenlő, az $N$ billentyűvel nemleges választ. Ha az almenü további menüpontokat tartalmaz, a főmenüből már ismert módon választhat közülük a felhasználó. (Csak a pont utáni részt szükséges beütnie.) Néhány esetben a program koordinátákat kér, ekkor egy szóközzel elválasztva kell megadni az $x$, majd az $y$ irányú koordinátát. A program itt egy-egy egész számot vár.\\
\par

A futások kimenete egy külső log fájlba fog kerülni, a felhasználó a konzolon keresztül csak arról fog visszajelzést kapni, hogy a végrehajtás részleteit a program kiírta ebbe a fájlba.\\
\par

A kimeneti fájlban az egyes futások elején megtalálható lesz a folyamat ami a konzolon a futtatás előtt lezajlott, valamint ezután az adott futás eredménye, azaz a meghívott függvények nevei. Ezek formátuma a következő lesz: $\left[ :ClassName\right].functionName(\left[ param \right]^*)$, ahol a $ClassName$ az adott osztály nevét, a $functionName$ a hívott függvény nevét, a $param$ kulcsszavak pedig egy-egy paramétert jelölnek. Felüldefiniált függvények esetén mindig a felüldefiniáló osztály neve íródik ki.

