
\subsection{Robot}

A robot objektum felelős azért, hogy tárolja a robot saját pozícióját és sebességét. Ez az objektum képes a pályán olajfoltokat illetve ragacsfoltokat elhelyezni, valamint minden robot tisztában van azzal hogy ezekből mekkora készlet áll még rendelkezésére. Ezen kívül a robotok egyenként ismerik a legfontosabb tulajdonságukat: hogy mekkora távolságot tettek meg már a játék során.

\subsection{Coord}

A Coord objektum Descartes-koordinátákat tartalmaz és visszaadja azokat. Ezen koordináták alapján valósul meg a játék több belső mechanizmusa.

\subsection{Cell}

A cellák tárolnak egy cellBehaviour interfészű objektumot ami megadja, hogy azok most éppen olajfoltosak, ragacsfoltosak, vagy üresek-e, - azaz egyik folt sem található meg rajtuk. A cellák összessége eredményezi a versenyzésre kijelölt pályát. A cellák szabályos sokszög alakúak, minden robot rendelkezik egy kezdőpozícióval, ami egy cellát jelent. A cellák ismerik középpontjuk koordinátáit.

\subsection{Map}

A Map osztály a cellák összességéből tevődik össze, ezen az objektumon versenyeznek a robotok valójában. A pálya bármilyen alakot felvehet majd. A Map tisztában van azzal, hogy a pályán hol van cella, s hol nincs. (Azaz, hogy melyik koordinátára ugorva marad életben a robot és melyik koordinátára ugorva hal meg.) A Map tudja megadni az adott koordinátájú cella szomszédos, üresen álló celláit is. Erre akkor van szükség, ha két robot egyszerre szeretne ugyanarra a cellára ugrani.

\subsection{Game}

A Game objektum köti össze a Robot és a Map objektumok működését és biztosítja majd a kapcsolatot a vezérlés felé. Képes beállítani a játék elején a kezdőértékeket. A játékhoz hozzá tud adni robotokat, el tudja távolítani azokat, illetve a léptetést is ez az objektum végzi, ami a játék körökre osztott mivoltát adja. Amennyiben ütközés van a pályán, észleli és lekezeli azokat. 