% Szglab4
% ===========================================================================
%
\section{Napló}

\begin{naplo}

\bejegyzes
{2015.02.25.~18:00} % Kezdet
{3 óra} % Időtartam
{Pilinszki-Nagy} % Résztvevők
{Tevékenység: A követelmények és szótár javítása a konzultációs utasítások alapján.} % Leírás

\bejegyzes
{2015.02.25.~22:00}
{3 óra}
{Gema}
{Tevékenység: Kezdeti statikus modell felvázolása, kommunikációs platform problémájára megoldás keresése}

\bejegyzes
{2015.02.26.~22:00}
{2 óra}
{Gema, Kemény}
{Tevékenység: Statikus modell korrigálása}

\bejegyzes
{2015.02.27.~15:00} % Kezdet
{5 óra} % Időtartam
{Gema, Kemény} % Résztvevők
{Tevékenység: Statikus modell végleges felépítése.} % Leírás

\bejegyzes
{2015.02.27.~15:30} % Kezdet
{0,5 óra} % Időtartam
{Gema, Juszt, Kemény, Pilinszki-Nagy, Somogyi} % Résztvevők
{Megbeszélés: Feladatok kiosztása a HipChaten keresztül. Döntés: Juszt, Pilinszki-Nagy, Somogyi: 3.1; Gema, Kemény: 3.2; szekvencia diagrammok az elkészült fejezetek alapján.} % Leírás

\bejegyzes
{2015.02.27.~16:00} % Kezdet
{2 óra} % Időtartam
{Juszt, Pilinszki-Nagy, Somogyi} % Résztvevők
{Tevékenység: 3.1, objektum katalógus elkészítése} % Leírás

\bejegyzes
{2015.02.27.~17:30} % Kezdet
{1 óra} % Időtartam
{Somogyi} % Résztvevők
{Tevékenység: Napló vezetése, előző heti feladatok javítása.} % Leírás

\bejegyzes
{2015.02.27.~20:45} % Kezdet
{2 óra} % Időtartam
{Gema, Kemény, Somogyi} % Résztvevők
{Meeting: megbeszélés Google Hangout-on keresztül. Témája: az elkészült statikus modell ismertetése a kollégákkal, a szekvencia diagrammok felépítésének megtervezése.} % Leírás

\bejegyzes
{2015.02.28.~16:00}
{2 óra}
{Gema, Kemény}
{Tevékenység: A dokumentáció 3.2-es pontjának (Osztályok leírása) elkészítése}

\bejegyzes
{2015.03.01.~09:00}
{5 óra}
{Juszt}
{Tevékenység: A rám kiosztott szekvencia diagrammok elkészítése}


\end{naplo}

