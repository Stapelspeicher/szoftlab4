
\subsection{Cell}
\begin{itemize}
\item Felelősség\\
Az osztály felelőssége a játék celláinak a reprezentációja. Ebben az osztályban tároljuk az adott cella középpontjának koordinátáit a megfelelő koordinátarendszerben, illetve ez az osztály tárolja a cella viselkedését is, ami megszabja, hogy hogyan hat a robotra, amelyik rálép.

\item Attribútumok
	\begin{itemize}
		\item center: A cella középpontjának a koordinátáit tárolja
		\item behaviour: A cella viselkedését tárolja
	\end{itemize}
\item Metódusok
	\begin{itemize}
		\item Cell(Coord center): Konstruktor, beállítja a cella középpontjának koordinátáit. A koordinátákat ahhoz fogunk használni, hogy meghatározzuk, hogy egy adott robot éppen melyik cellában van.
		\item Coord getCenter(): Visszaadja a cella középpontjának koordinátáit.
		\item void setBehaviour(CellBehaviour behaviour): Beállítja egy CellBehaviour interfészű objektumon keresztül, hogy mi történjen egy robottal, ha az adott cellára lép. (semmi, olaj hatása, ragacs hatása)
		\item CellBehaviour getBehaviour(): Visszaadja a cellára jellemző viselkedést tartalmazó objektumot, ami tartalmazza, hogy mi történik egy robottal ha az adott cellára lép.
	\end{itemize}
\end{itemize}

\subsection{CellBehaviour}
\begin{itemize}
	\item Felelősség\\
	Egy interfész, aminek használatával leírható egy cella viselkedése, azaz hogy mi történjen egy robottal ha az adott viselkedéssel ellátott cellára lép.
	\item Metódusok
	\begin{itemize}
		\item void stepOn(Robot): Ebben a metódusban lép interakcióba a viselkedés a paraméterként kapott robottal, azaz alkalmazza a viselkedést a robotra.
	\end{itemize}
\end{itemize}

\subsection{CellBehaviourProvider}
\begin{itemize}
	\item Felelősség\\
	Egy segédosztály, a mindegyik viselkedésből tárol egyet-egyet és elkérhető tőle ezen objektumok referenciája. Ez az optimalizációt szolgálja, mivel a viselkedést leíró objektumok állapot nélküliek és így elég belőlük egyet-egyet tárolni.
	\item Attribútumok
	\begin{itemize}
		\item plain: Statikus attribútum, egy olyan cella viselkedését tárolja, amin nincs se ragacs, se olajfolt.
		\item oily: Statikus attribútum, egy olyan cella viselkedését tárolja, amin olajfolt van.
		\item sticky: Statikus attribútum, egy olyan cella viselkedését tárolja, amin ragacs van.
	\end{itemize}
	\item Metódusok
	\begin{itemize}
		\item PlainCellBehaviour getPlain(): Egy, a plain attribútumra mutató referenciát ad vissza.
		\item OilyCellBehaviour getOily(): Egy, az oily attribútumra mutató referenciát ad vissza.
		\item StickyCellBehaviour getSticky(): Egy, a sticky attribútumra mutató referenciát ad vissza.		
	\end{itemize}
\end{itemize}

\subsection{Coord}
\begin{itemize}
	\item Felelősség\\
	Ez az osztály pontok Descartes-koordinátáit tárolja.
	\item Attribútumok
	\begin{itemize}
		\item x: A pont X koordinátája.
		\item y: A pont Y koordinátája.
	\end{itemize}
	\item Metódusok
	\begin{itemize}
		\item Coord(double x, double y): Konstruktor, paraméterként a pont X és Y koordinátáit kapja meg és állítja be a megfelelő attribútumnak.
		\item double getX(): A tárolt pont X koordinátáját adja vissza.
		\item double getY(): A tárolt pont Y koordinátáját adja vissza.
	\end{itemize}
\end{itemize}


\subsection{Game}
\begin{itemize}
	\item Felelősség\\
	Az osztály felelőssége a játék logikájához tartozó fő objektumok karbantartása és a játék léptetése. Ennek megfelelően rendelkezik az összes robottal, illetve a játék térképével. Ebből az osztályból minden játékhoz pontosan egy objektum tartozik.
	\item Attribútumok
	\begin{itemize}
		\item robots: Egy Map, ami a játékban található robotokat tárolja a nevükkel (játékosok neveivel) azonosítva őket.
		\item map: Az aktuális játék pályáját tartalmazó attribútum. Ezen a pályán lépkednek a robotok.
		\item rounds: Ebben az attribútumban a hátralevő körök számát tartja nyilván.
	\end{itemize}
	\item Metódusok
	\begin{itemize}
		\item Game(GameSettings s): Az osztály konstruktora, ez a függvény állítja be a kezdeti értékeket, és generálja le a robotokat a paraméterként kapott GameSettings objektum alapján. Ezen kívül meghívja a Map konstruktorát.
		\item void step(): Ez a függvény lépteti a játékot, itt mozognak a robotok, és itt kezdeményezzük az ütközések feloldását.
		\item Map<String, RobotController> getRobotControllers(): ez a függvény adja vissza a robotok irányító interfészét. A lényege az, hogy a játékban lévő robotokból csak annyi látszódjon kifele, amennyi minimálisan szükséges.
		\item void terminate(): Ez a függvény fejezi be a játékot, ha a játékos kézzel állította le azt. Ilyenkor nem hirdet győztest.
	\end{itemize}
\end{itemize}

\subsection{GameMap}
\begin{itemize}
	\item Felelősség\\
	A játék pályáját tárolja, és az interakciót biztosítja a pálya és a robotok, valamint a pálya és a Game osztály között. Ezen kívül a Descartes-koordináták cellaindexszé alakítását is ez az osztály végzi el.
	\item Attribútumok
	\begin{itemize}
		\item cells: Egy mátrix, ami a pálya celláit tárolja (és indexeli) Cell objektumok formájában.
	\end{itemize}
	\item Metódusok
	\begin{itemize}
		\item GameMap(File f): Az osztály konstruktora, betölti a pályát a paraméterként kapott fájlból a cells mátrixba.
		\item List<Coord> getFreeNeighbouring(List<Coord> robots, Coord dest): A metódus feladata visszaadni a dest paraméterként kapott pontnak megfelelő cella szomszédságában levő szabad cellákat. Szabad cella alatt azokat a cellákat értjük, amiken nincs se folt, se robot. A robots paraméter a robotok pozíciójának ismeretéhez szükséges.
		\item void addBehaviour(Coord c, CellBehaviour b): A c paraméternek megfelelő cellát ellátja a b paraméterként megkapott viselkedéssel. Azaz ezzel a metódussal megadható, hogy mi történjen egy robottal, ha a c paraméter által meghatározott cellára lép.
		\item bool isOnMap(Coord): Visszaadja, hogy a paraméterként megkapott koordináták által meghatározott pont a pályán van-e.
		\item void interactWith(Robot r): Interakcióba lép a paraméterként megkapott robottal, azaz a cella viselkedését (olaj vagy ragacs hatása) alkalmazza a robotra, valamint korrigálja a robot pozícióját, hogy mindig egy cella középpontjában álljon.
		\item Coord getRandomFreeCoord(): A robotok kezdeti elhelyezéséhez visszaadja egy olyan cella koordinátáit, amiben nincsen ragacs vagy olajfolt.
	\end{itemize}
\end{itemize}

\subsection{GameSettings}
\begin{itemize}
	\item Felelősség\\
	Az osztály felelőssége az, hogy azokat az információkat amik egy új játék elkezdéséhez szükségesek egységbe zárja, és így átadható legyen át a Game osztály konstruktorának, ami elvégzi az inicializálást.
	\item Attribútumok
	\begin{itemize}
		\item mapFile: A pályát tartalmazó File-ra mutat.
		\item initialSticky: Tárolja, hogy a robotok mennyi ragaccsal kezdjék a játékot.
		\item initialOily: Tárolja hogy a robotok hány olajfolttal kezdjék a játékot.
		\item rounds: Tárolja, hogy a játék hány körből álljon.
		\item robotNames: Ebben a listában adjuk át a robotok neveit, és ez implikálja azt is, hogy hány robotot akarunk létrehozni a játékban.
	\end{itemize}
	\item Metódusok
	\begin{itemize}
		\item void setMapFile(File): a mapFile attribútumot beállító metódus.
		\item File getMapFile(): A mapFile attribútum értékét visszaadó metódus.
		\item void setInitialSticky(int): Az initialSticky attribútumot beállító metódus.
		\item int getInitialSticky(): Az initialSticky attribútum értékét visszaadó metódus.
		\item void setInitialOily(int): Az initialOily attribútumot beállító metódus.
		\item int getinitialOily(): Az initialOily attribútum értékét visszaadó metódus.
		\item void setRounds(int): A rounds attribútumot beállító metódus.
		\item int getRounds(): A rounds attribútum értékét visszaadó metódus.
		\item void setRobotNames(int): A robotNames attribútumot beállító metódus.
		\item List<String> getRobotNames(): A robotNames attribútum értékét visszaadó metódus.
	\end{itemize}
\end{itemize}

\subsection{OilyCellBehaviour}
\begin{itemize}
	\item Felelősség\\
	Egy olyan cella viselkedését írja le, amin olajfolt van.
	\item Interfészek\\
	CellBehaviour
	\item Metódusok
	\begin{itemize}
		\item void stepOn(Robot): A paraméterként kapott robottal lép interakcióba, jelen esetben mivel egy olajos van szó, megtiltja számára, hogy a sebessége módosuljon az adott körben.
	\end{itemize}
\end{itemize}

\subsection{PlainCellBehaviour}
\begin{itemize}
	\item Felelősség\\
	Egy olyan cella viselkedését írja le, amin nincs se ragacs, se olajfolt.
	\item Interfészek\\
	CellBehaviour
	\item Metódusok
	\begin{itemize}
		\item void stepOn(Robot): A paraméterként kapott robottal lép interakcióba, jelen esetben mivel egy ragacs és olajfolt nélküli celláról van szó, nem tesz vele semmit.
	\end{itemize}
\end{itemize}

\subsection{Robot}
\begin{itemize}
	\item Felelősség\\
	A játékos által mozgatható robotok osztálya, ez az osztály tartalmazza a robotok összes logikáját, illetve tulajdonságát. Az osztály felelőssége az, hogy nyilvántartsa egy adott játékosokhoz tartozó robotok tulajdonságait.
	\item Interfészek\\
	RobotController
	\item Attribútumok
	\begin{itemize}
		\item distance: A robot által eddig megtett távolság.
		\item oilyNum: A robot rendelkezésére álló olajfoltok száma.
		\item stickyNum: A robot rendelkezésére álló ragacsok száma.
		\item velocity: A robot sebessége koordinátákként tárolva.
		\item position: A robot helyzete a pályán koordinátákként tárolva.
		\item map: A játék térképe, a foltok letételéhez szükséges. (Statikus attribútum)
	\end{itemize}
	\item Metódusok
	\begin{itemize}
		\item Robot(int oily, int sticky) A robot konstruktora, ami beállítja azt, hogy a robot mennyi ragaccsal illetve olajfolttal rendelkezzen.
		\item double getDistance(): Visszaadja a robot által eddig megtett út hosszát.
		\item int getOilyNum(): Visszaadja hogy még hány olajfoltja van a robotnak.
		\item int getStickyNum(): Visszaadja hogy még hány ragacsa van a robotnak.
		\item void addVelocity(Coord): A megkapott koordinátákat hozzáadja a sebességhez
		\item Coord getVelocity() : A robot sebességvektorának koordinátáit adja vissza.
		\item void setPosition(Coord): Beállítja a robot pozícióját a megkapott koordinátákra.
		\item Coord getPosition(): Visszaadja a robot aktuális pozícióját
		\item void placeSticky(): Letesz egy ragacsot a robot aktuális pozicióján.
		\item void placeOily(): Letesz egy olajfoltot a robot aktuális pozícióján.
		\item void setOily(): Beállítja, hogy a robot legyen olajos, azaz ne változhasson a sebességvektora az adott körben.
		\item void jump(): A robot pozíciójához hozzáadja az aktuális sebességét, azaz lépteti a robotot.
		\item void setMap(GameMap map): Beállítja a Robot osztálynak a pályát amelyen a játék zajlani fog, ez a csapdák lerakása miatt szükséges. (Statikus metódus)
	\end{itemize}
\end{itemize}

\subsection{RobotController}
\begin{itemize}
	\item Felelősség\\
	A Robot osztálynak biztosít egy interfészt csak annyi metódust láttatva a Robot osztály valós interfészéből kifele, amennyi valóban szükséges.
	\item Metódusok
	\begin{itemize}
		\item double getDistance(): Visszaadja a robot által eddig megtett út hosszát.
		\item int getOilyNum(): Visszaadja hogy még hány olajfoltja van a robotnak.
		\item int getStickyNum(): Visszaadja hogy még hány ragacsa van a robotnak.
		\item void addVelocity(Coord): A megkapott koordinátákat hozzáadja a sebességhez
		\item Coord getVelocity() : A robot sebességvektorának koordinátáit adja vissza.
		\item void placeSticky(): Letesz egy ragacsot a robot aktuális pozicióján.
		\item void placeOily(): Letesz egy olajfoltot a robot aktuális pozícióján.
	\end{itemize}
\end{itemize}

\subsection{StickyCellBehaviour}
\begin{itemize}
	\item Felelősség\\
	Egy olyan cella viselkedését írja le, amin ragacs van.
	\item Interfészek\\
	CellBehaviour
	\item Metódusok
	\begin{itemize}
		\item void stepOn(Robot): A paraméterként kapott robottal lép interakcióba, jelen esetben mivel egy ragacsos celláról van szó, megfelezi a sebességét.
	\end{itemize}
\end{itemize}
