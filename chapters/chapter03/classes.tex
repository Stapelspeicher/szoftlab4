
\subsection{Osztály1}
\begin{itemize}
\item Felelősség\\
\comment{Mi az osztály felelőssége. Kb 1 bekezdés.}
\item Ősosztályok\\
\comment{Mely osztályokból származik (öröklési hierarchia)\newline
Legősebb osztály $\rightarrow$ Ősosztály2 $\rightarrow$ Ősosztály3...}
\item Interfészek\\
\comment{Mely interfészeket valósítja meg.}
\item Attribútumok\\
\comment{Milyen attribútumai vannak}
	\begin{itemize}
		\item attribútum1: attribútum jellemzése: mire való
		\item attribútum2: attribútum jellemzése: mire való
	\end{itemize}
\item Metódusok\\
\comment{Milyen publikus metódusokkal rendelkezik. Metódusonként egy-három mondat arról, hogy a metódus mit csinál.}
	\begin{itemize}
		\item int foo(Osztály3 o1, Osztály4 o2): metódus leírása
		\item int bar(Osztály5 o1): metódus leírása
	\end{itemize}
\end{itemize}

\subsection{Osztály2}
\begin{itemize}
\item Felelősség\\
\comment{Mi az osztály felelőssége. Kb 1 bekezdés.}
\item Ősosztályok\\
\comment{Mely osztályokból származik (öröklési hierarchia)\newline
Legősebb osztály $\rightarrow$ Ősosztály2 $\rightarrow$ Ősosztály3...}
\item Interfészek\\
\comment{Mely interfészeket valósítja meg.}
\item Attribútumok\\
\comment{Milyen attribútumai vannak}
	\begin{itemize}
		\item attribútum1: attribútum jellemzése: mire való
		\item attribútum2: attribútum jellemzése: mire való
	\end{itemize}
\item Metódusok\\
\comment{Milyen publikus metódusokkal rendelkezik. Metódusonként egy-három mondat arról, hogy a metódus mit csinál.}
	\begin{itemize}
		\item int foo(Osztály3 o1, Osztály4 o2): metódus leírása
		\item int bar(Osztály5 o1): metódus leírása
	\end{itemize}
\end{itemize}
