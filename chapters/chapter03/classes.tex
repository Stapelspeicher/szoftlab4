
\subsection{Cell}
\begin{itemize}
\item Felelősség\\
Az osztály felelőssége a játék celláinak a reprezentációja. Ebben az osztályban tároljuk az adott cella koordinátáit a megfelelő koordináta rendszerben, illetve ez az osztály tartalmazza a Cella viselkedését is, ami megszabja hogy hogyan hat a robotra, amelyik rálép.

\item Ősosztályok\\
\comment{Mely osztályokból származik (öröklési hierarchia)}\newline
Object $\rightarrow$ Cell
\item Attribútumok\\
\comment{Milyen attribútumai vannak}
	\begin{itemize}
		\item center: A cella közepének a koordinátáját tárolja
		\item behaviour: A cella viselkedését tárolja
	\end{itemize}
\item Metódusok\\
\comment{Milyen publikus metódusokkal rendelkezik. Metódusonként egy-három mondat arról, hogy a metódus mit csinál.}
	\begin{itemize}
		\item void setCenter(Coord c): Beállítja a cella közepének koordinátáit. Amiket ahhoz fogunk használni, hogy megmondjuk hogy a robot éppen melyik cellában van.
		\item Coord getCenter(): Vissza lehet kérni, hogy hol van a cella közepe.
		q
	\end{itemize}
\end{itemize}

\subsection{Game}
\begin{itemize}
	\item Felelősség\\
	Az osztály felelőssége a Játékban található fő objektumok karban tartása és a játék léptetése, ehhez megfelelően rendelkezik az összes robottal, illetve a játék térképével. Ebből az osztályból minden játékhoz pontosan egy objektum tartozik.
	\item Ősosztályok\\
	Object $\rightarrow$ Game
	\item Attribútumok\\
	\begin{itemize}
		\item robots: Egy Map, ami a játékban található robotokat tárolja a nevükkel együtt.
		\item map: Az aktuális játék pályáját tartalmazó attribútum. Ezen a pályán lépkednek a robotok.
		\item rounds: Ebben az attribútumban köröket tartja nyilván.
	\end{itemize}
	\item Metódusok\\
	\begin{itemize}
		\item Robot[] checkCollission(Robot): Megnézi hogy az aktuális robottal ütközik-e másik robot. Ha igen, akkor visszaadja azoknak a Robotnoknak a Listáját, akik ütköznek.
		\item void resolveCollision(Robot[]): Kap egy listát olyan robotokról amik ütköznek, és eldönti, hogy mi legyen velük. Ha van elég szabad cella, akkor szétdobálja őket, ha nincs akkor pedig mind meghalnak.
		\item void kill(Robot): Eltávolítja a játékból a megkapott Robotot.
		\item Coord[] collectRobotPositions(): Visszaadja a játékban lévő összes Robot pozicióját.
		\item Game(Gamesettings): Az osztály konstruktora, ez a függvény állítja be a kezdeti értékeket, és generálja le a Robotokat a paraméterként kapott GameSettings objektum alapján.
		\item void step(): Ez a függvény lépteti a játékot, itt mozognak a robotok, és itt kezdeményezzük az ütközések feloldását.
		\item Map<String, RobotController> getRobotControllers(): ez a függvény Adja vissza a Robotok irányító interfacet, a lényege az, hogy a játékban lévő robotokból csak annyi látszódjon kifele, amennyi minimálisan szükséges.
		\item void terminate(): Ez a függvény fejezi be a játékot, ha a játékos kézzel állította le a játékot, akkor nem hirdet győztest.
	\end{itemize}
\end{itemize}


\subsection{GameSettings}
\begin{itemize}
	\item Felelősség\\
	Az osztály felelőssége az, hogy azokat az információkat amik egy új játék elkezdéséhez szükségesek egységbe zárja, és így adja át a Game osztály konstruktorának, ami elvégzi az inicializálást.
	\item Ősosztályok\\
		Object $\rightarrow$ GameSettings
	\item Attribútumok\\
	\begin{itemize}
		\item mapFile: A pályát tartalmazó File-ra mutat.
		\item initialSticky: tárolja, hogy a robotok mennyi ragaccsal kezdjenek.
		\item initialOily: tárolja hogy a robotok hány olajfolttal kezdjenek.
		\item rounds: tárolja, hogy a játék hány körből álljon.
		\item robotNames: ebben a listában adjuk át a robotok neveit, és ez implikálja azt is, hogy hány robotot akarunk létrehozni a játékban.
	\end{itemize}
	\item Metódusok\\
	\begin{itemize}
		\item void setMapFile(File): a mapFile attribútumot beállító metódus.
		\item File getMapFile(): A mapFile értékét visszaadó metódus.
		\item void setInitialSticky(int): az initialSticky attribútumot beállító metódus.
		\item int getInitialSticky(): Az initialSticky értékét visszaadó metódus.
		\item void setInitialOily(int): Az initialOily attribútumot beállító metódus.
		\item int getinitialOily(): Az initialOily értékét visszaadó metódus.
		\item void setRounds(int): A rounds attribútumot beállító metódus.
		\item int getRounds(): A rounds értékét visszaadó metódus.
		\item void setRobotNames(int): A robotNames attribútumot beállító metódus.
		\item List<String> getRobotNames(): A robotNames értékét visszaadó metódus.
	\end{itemize}
\end{itemize}

\subsection{Osztály2}
\begin{itemize}
\item Felelősség\\
\comment{Mi az osztály felelőssége. Kb 1 bekezdés.}
\item Ősosztályok\\
\comment{Mely osztályokból származik (öröklési hierarchia)\newline
Legősebb osztály $\rightarrow$ Ősosztály2 $\rightarrow$ Ősosztály3...}
\item Interfészek\\
\comment{Mely interfészeket valósítja meg.}
\item Attribútumok\\
\comment{Milyen attribútumai vannak}
	\begin{itemize}
		\item attribútum1: attribútum jellemzése: mire való
		\item attribútum2: attribútum jellemzése: mire való
	\end{itemize}
\item Metódusok\\
\comment{Milyen publikus metódusokkal rendelkezik. Metódusonként egy-három mondat arról, hogy a metódus mit csinál.}
	\begin{itemize}
		\item int foo(Osztály3 o1, Osztály4 o2): metódus leírása
		\item int bar(Osztály5 o1): metódus leírása
	\end{itemize}
\end{itemize}
