% Szglab4
% ===========================================================================
%
\chapter{Szkeleton tervezése}

\thispagestyle{fancy}

\section{A szkeleton modell valóságos use-case-ei}

\subsection{Use-case diagram}

\begin{figure}[h]
\begin{center}
\includegraphics[width=17cm]{chapters/chapter05/use_case.png}
\caption{x}
\label{fig:SzkeletonUseCase}
\end{center}
\end{figure}

\subsection{Use-case leírások}



\usecase{Játék indítása}{Az egyik játékos elindítja a versenyt.}{Felhasználó}{A program menüjéből a felhasználó ki tudja választani a Start game funkciót.}

\usecase{Kilépés a játékból}{A felhasználók ki tudnak lépni a játékból.}{Felhasználó}{A játékosok a verseny közben ki tudnak lépni a játékból.}

\usecase{Sebesség módosítása}{A robot sebesség vektora módosul.}{Játékos, Robot, Ragacsfolt}{Egy játékos megváltoztatja saját robotjának a sebességét,  az elérni kívánt cella irányába.}

\usecase{Olajfolt elhelyezése}{A robot egy olajfoltot hagy maga után.}{Játékos, Robot}{A játékos utasítására a robot elhelyez egy olajfoltot azon a cellán, amelyiken épp áll.}

\usecase{Ragacsfolt elhelyezése}{A robot egy ragacsfoltot hagy maga után.}{Játékos, Robot}{A játékos utasítására a robot elhelyez egy ragacsfoltot azon a cellán, amelyiken épp áll.}

\usecase{Kör vége}{Léptetés egy körrel.}{Játékos}{Ha minden játékos megtette a lépéseit, és kiadta a kívánt utasításokat a robotoknak, akkor a kör befejeződik, s egy új kezdődik.}

\usecase{Megsemmisülés}{Egy robot megsemmisülése.}{Robot}{Ha két robot ütközik, s nincs megfelelő cella számukra, akkor a két robot meghal és kiesik a játékból.}

\usecase{Olajfoltra lépés}{A robot rálép egy olajfoltra.}{Robot}{Egy robot egy olajfoltos cellára kénytelen lépni, ezáltal a következő körben a sebessége nem módosítható.}

\usecase{Ragacsfoltra lépés}{A robot rálép egy ragacsfoltra.}{Robot}{Egy robot egy ragacsfoltos cellára kénytelen lépni, ezáltal a sebessége megfeleződik.}

\usecase{Ütközés}{A robotok ütköznek.}{Robot}{Két robot azonos cellára lép, így összeütköznek.}


\section{A szkeleton kezelői felületének terve, dialógusok}
\comment{A szkeleton által elfogadott bemenetek , valamint a szöveges konzolon megjelenő kimenetek. A kiemenet formátuma olyan kell legyen, ami alapján a működés összevethető a korábbi szekvencia-diagramokkal.}

\section{Szekvencia diagramok a belső működésre}
\comment{A szkeletonban implementált szekvenciadiagramok. Tipikusan egy use-case egy diagram. Ezek megegyezhetnek a korábban specifikált diagramokkal, de az egyes életvonalakat (lifeline) egyértelműen a szkeletonban példányosított objektumokhoz kell tudni kötni. Azt kell megjeleníteni, hogy a szkeletonban létrehozott objektumok egymással hogyan fognak kommunikálni.}

\begin{figure}[!htbp]
	\begin{center}
		\includegraphics[width=13cm]{./chapters/chapter05/changevelocity.png}
		\caption{Sebesség módosítása}
	\end{center}
\end{figure}

\clearpage

\begin{figure}[!htbp]
	\begin{center}
		\includegraphics[width=13cm]{./chapters/chapter05/placeoilysequence.png}
		\caption{Olaj lerakása}
	\end{center}
\end{figure}

\begin{figure}[!htbp]
	\begin{center}
		\includegraphics[width=13cm]{./chapters/chapter05/placestickysequence.png}
		\caption{Ragacs lerakása}
	\end{center}
\end{figure}

\clearpage

\begin{figure}[!htbp]
	\begin{center}
		\includegraphics[width=18cm]{./chapters/chapter05/stepsequence.png}
		\caption{Lépés}
	\end{center}
\end{figure}

\clearpage

\begin{figure}[!htbp]
	\begin{center}
		\includegraphics[width=18cm]{./chapters/chapter05/stepoilysequence.png}
		\caption{Lépés olajra}
	\end{center}
\end{figure}

\clearpage

\begin{figure}[!htbp]
	\begin{center}
		\includegraphics[width=18cm]{./chapters/chapter05/stepstickysequence.png}
		\caption{Lépés ragacsra}
	\end{center}
\end{figure}




\section{Kommunikációs diagramok}
\comment{A szkeletonban, az egyes szkeleton-use-case-ek futása során létrehozott objektumok és kapcsolataik bemutatására szolgáló diagramok. Ezek alapján valósítják meg a szkeleton fejlesztői az inicializáló kódrészleteket.}

\begin{figure}[!htbp]
	\begin{center}
		\includegraphics[width=13cm]{./chapters/chapter05/placetrapobject.png}
		\caption{Olajfolt / ragacsfolt lerakása}
	\end{center}
\end{figure}


\begin{figure}[!htbp]
	\begin{center}
		\includegraphics[width=13cm]{./chapters/chapter05/stepobject.png}
		\caption{Lépés, sebesség módosítása}
	\end{center}
\end{figure}
