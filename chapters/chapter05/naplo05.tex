% Szglab4
% ===========================================================================
%
\section{Napló}

\begin{naplo}

\bejegyzes
{2015.03.13.~8:00} % Kezdet
{1,5 óra} % Időtartam
{Juszt, Kemény, Pilinszki-Nagy, Somogyi} % Résztvevők
{Meeting: heti feladatok megbeszélése. Döntés: Juszt, Kemény, Pilinszki-Nagy csinálja a use-case diagramokat, Somogyi a use-case leírásokat.} % Leírás

\bejegyzes
{2015.03.13.~15:00} % Kezdet
{3 óra} % Időtartam
{Kemény} % Résztvevők
{Tevékenység: uj statikus modell elkészítése.} % Leírás

\bejegyzes
{2015.03.13.~19:00} % Kezdet
{2,5 óra} % Időtartam
{Somogyi} % Résztvevők
{Tevékenység: use-case leírások.} % Leírás

\bejegyzes
{2015.03.13.~15:00} % Kezdet
{6 óra} % Időtartam
{Juszt} % Résztvevők
{Tevékenység: múlt heti hibás diagrammok javítása} % Leírás

\bejegyzes
{2015.03.14.~16:00} % Kezdet
{7 óra} % Időtartam
{Juszt} % Résztvevők
{Tevékenység: Új diagrammok elkészítése (szekvencia, objektum)} % Leírás

\bejegyzes
{2015.03.14.~18:00}
{1 óra}
{Gema}
{Tevékenység: Use-case leírások tanulmányozása, módosítási javaslatok összegyűjtése}

\bejegyzes
{2015.03.15.~19:00}
{5 óra}
{Gema}
{Tevékenység: A kiosztott szekvenciadiagramok elkészítése és a szkeleton kezelőfelületének specifikálása}

\bejegyzes
{2015.03.15.~18:00} % Kezdet
{3 óra} % Időtartam
{Kemény} % Résztvevők
{Tevékenység: szekvencia diagramok készítése.} % Leírás

\bejegyzes
{2015.03.15.~12:00} % Kezdet
{10 óra} % Időtartam
{Pilinszki-Nagy} % Résztvevők
{Tevékenység: diagramok készítése(szekvencia, objektum)} % Leírás

\end{naplo}

