

\usecase{Játék indítása}{Az egyik játékos elindítja a versenyt.}{Felhasználó}{A program menüjéből a felhasználó ki tudja választani a Start game funkciót.}

\usecase{Kilépés a játékból}{A felhasználók ki tudnak lépni a játékból.}{Felhasználó}{A játékosok a verseny közben ki tudnak lépni a játékból.}

\usecase{Sebesség módosítása}{A robot sebesség vektora módosul.}{Játékos, Robot, Ragacsfolt}{Egy játékos megváltoztatja saját robotjának a sebességét,  az elérni kívánt cella irányába.}

\usecase{Olajfolt elhelyezése}{A robot egy olajfoltot hagy maga után.}{Játékos, Robot}{A játékos utasítására a robot elhelyez egy olajfoltot azon a cellán, amelyiken épp áll.}

\usecase{Ragacsfolt elhelyezése}{A robot egy ragacsfoltot hagy maga után.}{Játékos, Robot}{A játékos utasítására a robot elhelyez egy ragacsfoltot azon a cellán, amelyiken épp áll.}

\usecase{Kör vége}{Léptetés egy körrel.}{Játékos}{Ha minden játékos megtette a lépéseit, és kiadta a kívánt utasításokat a robotoknak, akkor a kör befejeződik, s egy új kezdődik.}

\usecase{Megsemmisülés}{Egy robot megsemmisülése.}{Robot}{Ha két robot ütközik, s nincs megfelelő cella számukra, akkor a két robot meghal és kiesik a játékból.}

\usecase{Olajfoltra lépés}{A robot rálép egy olajfoltra.}{Robot}{Egy robot egy olajfoltos cellára kénytelen lépni, ezáltal a következő körben a sebessége nem módosítható.}

\usecase{Ragacsfoltra lépés}{A robot rálép egy ragacsfoltra.}{Robot}{Egy robot egy ragacsfoltos cellára kénytelen lépni, ezáltal a sebessége megfeleződik.}

\usecase{Ütközés}{A robotok ütköznek.}{Robot}{Két robot azonos cellára lép, így összeütköznek.}