
\subsection{Cél}
A megrendelő által meghatározott követelmények, a projekt alapvető felépítésének és funkcionalitásának ismertetése, körülírni ezek segítségével a projekt fejlesztésének menetét és működését. Az itt leírtaktól eltérni nem szabad, fejlesztés közben ezeket folyamatosan figyelembe kell majd venni.

\subsection{Szakterület}

Az elkészült szoftver egy számítógépes játék lesz, ebből adódóan a szórakoztató ipar igényeinek igyekszik megfelelni. Kiskorú felhasználók számára nyújt elsősorban kiváló időtöltési lehetőséget, de felnőttek is nagy örömet lelhetnek majd a program felhasználása során. A játék a műszaki beállítottságú vásárlókat célozza meg, leginkább nekik ajánlható a szoftver.

\subsection{Definíciók, rövidítések}


\noindent \textbf{architekturális kép:} A szoftver vázlatos belső felépítését szemléltető ábra.\\

\noindent\textbf{Eclipse:} Fejlesztőkörnyezet, amelyben a szoftver íródott, Java nyelvhez használható kiválóan.\\

\noindent\textbf{Enterprise Architect:} UML diagramok készítésére használható szoftver.\\

\noindent\textbf{Facebook:} A világ legnépszerűbb és legismertebb közösségi oldala.\\

\noindent\textbf{fejlesztőkörnyezet:} Olyan szoftver, amely lehetőleg egyszerűen alkalmazható fejlesztési munka végrehajtására, például az Eclipse.\\

\noindent\textbf{funkció:} A program működésének egy külön megfogalmazható része.\\

\noindent\textbf{Git:} Verziókezelőrendszer.\\

\noindent\textbf{Github:} Internetes szolgáltatás, amely a Git alapú verziókövetést könnyíti meg.\\

<<<<<<< HEAD
\noindent\textbf{LaTeX:} TeX-en alapuló szövegformázó rendszer, a dokumentáció készítéséhez használjuk.\\
=======
\noindent\textbf{LaTeX:} Szövegszerkesztő.\\
>>>>>>> origin/master

\noindent\textbf{PC:} Personal Computer, jelentése személyi számítógép.


\subsection{Hivatkozások}

A Budapesti Műszaki Egyetem mérnökinformatikus képzésének Szoftver labor 4 című tárgya:  \\
\url{https://www.iit.bme.hu/~szoftlab4/}\\

A LaTeX szövegszerkesztő használatához szükséges utasítások gyűjteménye:\\
\url{http://en.wikibooks.org/wiki/LaTeX}\\


\subsection{Összefoglalás}

 További fejezetek:\\

 2.2. Áttekintés - A szoftver bemutatása.\\

 2.3. Követelmények - A megrendelő és a szoftver követelményei, melyek alapján haladva a fejlesztés történik.\\

 2.4. Lényeges use-case-ek - A use-case-ek nevének felsorolása, rövid leírása, aktorok felsorolása, forgatókönyv.\\

 2.5. Szótár - A projekthez illetve szoftverhez kapcsolódó idegen, nem hétköznapi szavak gyűjteménye.\\

 2.6. Projekt terv - A munkafolyamat végrehajtásának előzetes terve.\\

 2.7. Napló. - A projekt fejlesztésének a lépéseit tartalmazza.