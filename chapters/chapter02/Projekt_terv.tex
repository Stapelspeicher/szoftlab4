%\subsection{Projekt terv}
\subsection{Csapat}
A csapat öt főből áll, próbálunk mindenkinek azonos nehézségű feladatokat találni, 
ami lehetőleg megfelel az egyéni preferenciáinak, nincsenek dedikált szerepek, mindenki foglalkozik mindennel.

\begin{center}
	\begin{tabular}{ | c | c | }
	\hline
		\multicolumn{2}{ | c | }{
			\textbf{Stapelspeicher csapat}} \\ \hline
		\textbf{Név} & 
		\textbf{Feladatkör} 
		\\ \hline \hline
		Kemény Károly & the boss \\ \hline
		Gema Barnabás &  cickány \\ \hline
		Juszt Ádám & the punctual \\ \hline
		Somogyi Gábor & the loud guy \\ \hline
		Pilinszki-Nagy Csongor & the manager \\ \hline
		
		
	\end{tabular}
\end{center}

\subsection{Kommunikáció}

Verziókezelésre a csapat a Git nevű verziókezelő rendszerre épülő Github platformot használja, azért mert ingyenes, elterjedt, és kiváló szolgáltatások épülnek köré. Így tartja nyilván a csapat a dokumentáció illetve a forráskód inkrementális változásait.\\

A csapat egy privát facebook csoportot használ fórum jelleggel, ide kerülnek ki, és innen kereshetőek vissza a közérdekű információk, például a megbeszélések időpontjai.\\

A különböző feladatrészek a megbeszéléseken kerülnek kiosztásra, amiket úgy tervezünk, hogy lehetőleg mindenki részt tudjon venni rajta személyesen. Ezen felül még folytatunk skype megbeszéléseket is.\\

\subsection{Használt Programok}

 Verziókezelésre Git programot használunk, a main repositoryt a Github szolgáltatja, amely egyszerűvé teszi a kollaborálást.\\

Fejlesztéshez a csapat Eclipset illetve IntelliJ IDEA-t használ.\\

Dokumentációra LaTeX-et használunk. A szerkesztő környezetként a TeXstudio nevű szoftver szolgál, illetve az egész dokumentáció a projekttel együtt egy repositoryban verziókezelés alatt áll.\\

\subsection{Mérföldkövek, Határidők}

\begin{center}
	\begin{tabular}{ | l | l | l | }
		\hline
		\textbf{Dátum} &
		\textbf{Feladat} &
		\textbf{Ellenőrzés}
		
		\\ \hline \hline
		febr. 23. & Követelmény, projekt, funkcionalitás & beadás
		\\ \hline 
		márc. 2. &	Analízis modell kidolgozása 1. & beadás
		\\ \hline 
		márc. 9. &	Analízis modell kidolgozása 2. & beadás
		\\ \hline 
		márc. 16. &	Szkeleton tervezése & beadás
		\\ \hline 		
		márc. 23. &	Szkeleton & beadás
		\\ \hline
		márc. 25. & Szkeleton & bemutató
		\\ \hline 
		márc. 30. &	Prototípus koncepciója & beadás
		\\ \hline 
		ápr. 7.	 & Részletes tervek & beadás
		\\ \hline 
		ápr. 20. &	Prototípus & beadás
		\\ \hline 
		ápr. 22. & Prototípus & bemutató
		\\ \hline
		ápr. 27. &	Grafikus felület specifikációja & beadás
		\\ \hline 	 
		máj. 11. &	Grafikus változat & beadás
		\\ \hline
		máj. 13. & Grafikus bemutató & bemutató
		\\ \hline
		máj. 15. &	Összefoglalás & beadás
		\\ \hline 
	\end{tabular}
\end{center}

A szkeleton változat célja annak bizonyítása, hogy az objektum és dinamikus modellek a definiált feladat egy modelljét alkotják. A szkeleton egy program, amelyben már valamennyi, a végső rendszerben is szereplő business objektum szerepel. Az objektumoknak csak az interfésze definiált. Valamennyi metódus az indulás pillanatában az ernyőre szöveges változatban kiírja a saját nevét, majd meghívja azon metódusokat, amelyeket a szolgáltatás végrehajtása érdekében meg kell hívnia. Amennyiben a metódusból valamely feltétel fennállása esetén hívunk meg más metódusokat, akkor a feltételre vonatkozó kérdést interaktívan az ernyőn fel kell tenni és a kapott válasz alapján kell a továbbiakban eljárni. A szkeletonnak alkalmasnak kell lenni arra, hogy a különböző forgatókönyvek és szekvencia diagramok ellenőrizhetők legyenek. Csak karakteres ernyőkezelés fogadható el, mert ez biztosítja a rendszer egyszerűségét.\\

A prototípus program célja annak demonstrálása, hogy a program elkészült, helyesen működik, valamennyi feladatát teljesíti. A prototípus változat egy elkészült program kivéve a kifejlett grafikus interfészt. A változat tervezési szempontból elkészült, az ütemezés, az aktív objektumok kezelése megoldott. A business objektumok - a megjelenítésre vonatkozó részeket kivéve - valamennyi metódusa a végleges algoritmusokat tartalmazza. A megjelenítés és működtetés egy alfanumerikus ernyőn követhető, ugyanakkor a megjelenítés fájlban is logolható, ezzel megteremtve a rendszer tesztelésének lehetőségét. Különös figyelmet kell fordítani az interfész logikájára, felépítésére, valamint arra, hogy az mennyiben tükrözi és teszi láthatóvá a program működését, a beavatkozások hatásait. \\

A grafikus (teljes) változat a prototípustól elvileg csak a kezelői felület minőségében különbözhet. Ennek
változatnak az értékelésekor a hangsúlyt sokkal inkább a megvalósítás belső szerkezetére, semmint a külalakra
kell helyezni.\\