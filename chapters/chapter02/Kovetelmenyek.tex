\subsection{Funkcionális követelmények}

% Azonosító, Leírás, Ellenőrzés, Prioritás, Forrás, Use-case, Komment
\begin{longtable}{| L{1cm} | L{4cm} | l | l | l | L{2cm} | l |}
\hline
\textbf{Azonosító}   & \textbf{Leírás} & \textbf{Ellenőrzés} & \textbf{Prioritás} & \textbf{Forrás} & \textbf{Use-case} & \textbf{Komment} \tabularnewline
\hline\hline
1.01 & Pálya építhető & bemutatás & fontos & megrendelő & Játékindítás & \tabularnewline
\hline
1.02 & Robot irányítható & bemutatás & fontos & megrendelő & Sebességvektor módosítása & \tabularnewline
\hline
1.03 & Egy robot mozgatható a pályán & bemutatás & fontos & megrendelő & Sebességvektor módosítása & \tabularnewline
\hline
1.04 & A robot által megtett távolság kiszámolható & bemutatás & alapvető & megrendelő & Lépés &  \tabularnewline
\hline
1.05 & A robot leesvén a pályáról kiesik & bemutatás & alapvető & megrendelő & Leugrás a pályáról &  \tabularnewline
\hline
1.06 & Ha teljesülnek a feltételek, a játék véget ér & bemutatás & fontos & megrendelő & Lépés &  \tabularnewline
\hline
1.07 & Játék vége után az eredmény kiíródik és új játék kezdhető & bemutatás & fontos & megrendelő & - &  \tabularnewline
\hline
1.08 & Foltok helyezhetők el a pályán & bemutatás & alapvető & megrendelő & Játékindítás & \tabularnewline
\hline
1.09 & Robotnak van saját foltkészlete & bemutatás & alapvető & megrendelő & Játékindítás &  \tabularnewline
\hline
1.10 & A robot el tudja helyezni a foltokat & bemutatás & alapvető & megrendelő & Ragacs hátrahagyása, Olaj hátrahagyása &  \tabularnewline
\hline
1.11 & Több robot mozoghat a pályán & bemutatás & fontos & megrendelő & - &  \tabularnewline
\hline
1.12 & A robotok egymás után jönnek & bemutatás & alapvető & megrendelő & Beállítások rögzítése &  \tabularnewline
\hline
1.13 & A játék közben elő lehet hívni egy menü ablakot & bemutatás & opcionális & csapat & - &  \tabularnewline
\hline
1.14 & A játék tesztelése kódokkal gyorsítható & bemutatás & opcionális & csapat & - &  \tabularnewline
\hline
\end{longtable}

\subsection{Erőforrásokkal kapcsolatos követelmények}

% Azonosító, Leírás, Ellenőrzés, Prioritás, Forrás, Komment
\begin{longtable}{| l | l | l | l | l | l |}
\hline
\textbf{Azonosító}   & \textbf{Leírás} & \textbf{Ellenőrzés} & \textbf{Prioritás} & \textbf{Forrás} & \textbf{Komment} \tabularnewline
\hline\hline
2.01 & Git & nincs & alapvető & csapat & Verziókezelés \tabularnewline
\hline
2.02 & GitHub & nincs & alapvető & csapat & Git tárhely \tabularnewline
\hline
2.03 & SourceTree & nincs & alapvető & csapat & Git GUI \tabularnewline
\hline
2.04 & Tex Studio & nincs & alapvető & csapat & Latex szerkesztő \tabularnewline
\hline
2.05 & JRE6 & nincs & fontos & megrendelő &  \tabularnewline
\hline
2.06 & Eclipse & nincs & alapvető & csapat & Java IDE  \tabularnewline
\hline
2.07 & Facebook & nincs & alapvető & csapat & Kapcsolattartás \tabularnewline
\hline
2.08 & Enterprise Architekt & nincs & alapvető & csapat & Modellező szotver \tabularnewline
\hline
\end{longtable}


\subsection{Átadással kapcsolatos követelmények}


\begin{longtable}{| L{1cm} | L{4cm} | l | l | l | l |}
\hline
\textbf{Azonosító}   & \textbf{Leírás} & \textbf{Ellenőrzés} & \textbf{Prioritás} & \textbf{Forrás} & \textbf{Komment} \tabularnewline
\hline\hline
3.01 & Szkeleton átadás & bemutatás & fontos & megrendelő & márc. 23. \tabularnewline
\hline
3.02 & Proto átadás & bemutatás & fontos & megrendelő & ápr. 20. \tabularnewline
\hline
3.03 & Grafikus átadás & bemutatás & fontos & megrendelő & máj. 11. \tabularnewline
\hline
3.04 & A programnak működnie kell a HSZK gépein & bemutatás & fontos & megrendelő &  \tabularnewline
\hline
\end{longtable}

\subsection{Egyéb nem funkcionális követelmények}


% Azonosító, Leírás, Ellenőrzés, Prioritás, Forrás, Komment



