\subsection{Funkcionális követelmények}


% Azonosító, Leírás, Ellenőrzés, Prioritás, Forrás, Use-case, Komment
\begin{longtable}{| l | l | l | l | l | l | l |}
\hline
\textbf{Azonosító}   & \textbf{Leírás} & \textbf{Ellenőrzés} & \textbf{Prioritás} & \textbf{Forrás} & \textbf{Use-case} & \textbf{Komment} \tabularnewline
\hline\hline
1.01 & Absztrakt pálya & bemutatás & fontos & megrendelő & Main use-case & komment \tabularnewline
\hline
1.02 & Robot irányító felület & bemutatás & fontos & megrendelő & Main use-case & komment \tabularnewline
\hline
1.03 & Egy robot mozog a pályán & bemutatás & fontos & megrendelő &  & komment \tabularnewline
\hline
1.04 & Távolság számítása & bemutatás & alapvető & megrendelő & Eredmény use-case & komment \tabularnewline
\hline
1.05 & A robot leesik a pályáról & bemutatás & alapvető & megrendelő &  & komment \tabularnewline
\hline
1.06 & Játék vége & bemutatás & fontos & megrendelő & Eredmény use-case & Megadott idő/kör után vagy ha a robot leesik a pályáról \tabularnewline
\hline
1.07 & Eredmény kiírása, új játék & bemutatás & fontos & megrendelő & Eredmény use-case & komment \tabularnewline
\hline
1.08 & A robot foltba megy& bemutatás & alapvető & megrendelő &  & komment \tabularnewline
\hline
1.09 & Robot foltkészlete & bemutatás & alapvető & megrendelő & Main use-case & komment \tabularnewline
\hline
1.10 & A robot felszedi a foltokat & bemutatás & opcionális & csapat &  & komment \tabularnewline
\hline
1.11 & A robot elhelyezi a foltokat & bemutatás & alapvető & megrendelő &  & komment \tabularnewline
\hline
1.12 & Több robot a pályán & bemutatás & fontos & megrendelő &  & komment \tabularnewline
\hline
1.13 & A robotok egymás után jönnek & bemutatás & alapvető & megrendelő &  & komment \tabularnewline
\hline
1.14 & Menürendszer & bemutatás & opcionális & csapat & Menu use-case & komment \tabularnewline
\hline
1.15 & Cheat-kódok & bemutatás & opcionális & csapat & Menu use-case & komment \tabularnewline
\hline
\end{longtable}

\subsection{Erőforrásokkal kapcsolatos követelmények}

% Azonosító, Leírás, Ellenőrzés, Prioritás, Forrás, Komment
\begin{longtable}{| l | l | l | l | l | l |}
\hline
\textbf{Azonosító}   & \textbf{Leírás} & \textbf{Ellenőrzés} & \textbf{Prioritás} & \textbf{Forrás} & \textbf{Komment} \tabularnewline
\hline\hline
2.01 & Git & nincs & alapvető & csapat & Verziókezelés \tabularnewline
\hline
2.02 & GitHub & nincs & alapvető & csapat & Git tárhely \tabularnewline
\hline
2.03 & SourceTree & nincs & alapvető & csapat & Git GUI \tabularnewline
\hline
2.04 & Tex Studio & nincs & alapvető & csapat & Latex szerkesztő \tabularnewline
\hline
2.05 & JRE6 & nincs & fontos & megrendelő &  \tabularnewline
\hline
2.06 & Eclipse & nincs & alapvető & csapat & Java IDE  \tabularnewline
\hline
2.07 & Facebook & nincs & alapvető & csapat & Kapcsolattartás \tabularnewline
\hline
2.08 & Enterprise Architekt & nincs & alapvető & csapat & Modellező szotver \tabularnewline
\hline
\end{longtable}


\subsection{Átadással kapcsolatos követelmények}


\begin{longtable}{| l | l | l | l | l | l |}
\hline
\textbf{Azonosító}   & \textbf{Leírás} & \textbf{Ellenőrzés} & \textbf{Prioritás} & \textbf{Forrás} & \textbf{Komment} \tabularnewline
\hline\hline
3.01 & Szkeleton átadás & bemutatás & fontos & megrendelő & márc. 23. \tabularnewline
\hline
3.02 & Proto átadás & bemutatás & fontos & megrendelő & márc. 23. \tabularnewline
\hline
3.03 & Grafikus átadás & bemutatás & fontos & megrendelő & márc. 23. \tabularnewline
\hline
3.04 & A programnak működnie kell a HSZK gépein & bemutatás & fontos & megrendelő & márc. 23. \tabularnewline
\hline
\end{longtable}

\subsection{Egyéb nem funkcionális követelmények}


% Azonosító, Leírás, Ellenőrzés, Prioritás, Forrás, Komment
\begin{longtable}{| l | l | l | l | l | l |}
\hline
\textbf{Azonosító}   & \textbf{Leírás} & \textbf{Ellenőrzés} & \textbf{Prioritás} & \textbf{Forrás} & \textbf{Komment} \tabularnewline
\hline\hline
4.01 & Jól nézzen ki a játék & nincs & opcionális & csapat & \tabularnewline
\hline
\end{longtable}

