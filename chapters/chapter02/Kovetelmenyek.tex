\subsection{Funkcionális követelmények}

\comment{Az alábbi táblázat kitöltésével készítendő. Dolgozzon ki követelmény azonosító rendszert! Az ellenőrzés módja szokásosan bemutatás és/vagy kiértékelés. Prioritás lehet alapvető, fontos, opcionális. Az alapvető követelmények nem teljesítése végzetes. Forrás alatt a követelményt előíró anyagot, szervezetet kell érteni. Esetünkben forrás lehet maga a csapat is, mikor ő talál ki követelményt. Use-case-ek alatt az adott követelményt megvalósító használati esete(ke)t kell megadni.}

% Azonosító, Leírás, Ellenőrzés, Prioritás, Forrás, Use-case, Komment
\begin{longtable}{| l | l | l | l | l | l | l |}
\hline
\textbf{Azonosító}   & \textbf{Leírás} & \textbf{Ellenőrzés} & \textbf{Prioritás} & \textbf{Forrás} & \textbf{Use-case} & \textbf{Komment} \tabularnewline
\hline\hline
1.01 & Absztrakt pálya & bemutatás & fontos & megrendelő & dunno & komment \tabularnewline
\hline
1.02 & Kezelő felület & bemutatás & fontos & megrendelő & dunno & komment \tabularnewline
\hline
1.03 & Egy robot mozog a pályán & bemutatás & fontos & megrendelő & dunno & komment \tabularnewline
\hline
1.04 & Távolság számítása & bemutatás & alapvető & megrendelő & dunno & komment \tabularnewline
\hline
1.05 & A robot leesik a pályáról & bemutatás & alapvető & megrendelő & dunno & komment \tabularnewline
\hline
1.06 & Játék vége & bemutatás & fontos & megrendelő & dunno & Megadott idő/kör után vagy ha a robot leesik a pályáról \tabularnewline
\hline
1.07 & Eredmény kiírása, új játék & bemutatás & fontos & megrendelő & dunno & komment \tabularnewline
\hline
1.08 & A robot foltba megy& bemutatás & alapvető & megrendelő & dunno & komment \tabularnewline
\hline
1.09 & A robot felszedi a foltokat & bemutatás & opcionális & csapat & dunno & komment \tabularnewline
\hline
1.10 & A robot elhelyezi a foltokat & bemutatás & alapvető & megrendelő & dunno & komment \tabularnewline
\hline
1.11 & Több robot a pályán & bemutatás & fontos & megrendelő & dunno & komment \tabularnewline
\hline
1.12 & A robotok egymás után jönnek & bemutatás & alapvető & megrendelő & dunno & komment \tabularnewline
\hline
1.13 & Menürendszer & bemutatás & opcionális & megrendelő & dunno & komment \tabularnewline
\hline
\end{longtable}

\subsection{Erőforrásokkal kapcsolatos követelmények}

\comment{A szoftver fejlesztésével és használatával kapcsolatos számítógépes, hardveres, alapszoftveres és egyéb architekturális és logisztikai követelmények}

% Azonosító, Leírás, Ellenőrzés, Prioritás, Forrás, Komment
\begin{longtable}{| l | l | l | l | l | l |}
\hline
\textbf{Azonosító}   & \textbf{Leírás} & \textbf{Ellenőrzés} & \textbf{Prioritás} & \textbf{Forrás} & \textbf{Komment} \tabularnewline
\hline\hline
... & ... & ... & ... & ... & ... \tabularnewline
\hline
\end{longtable}


\subsection{Átadással kapcsolatos követelmények}
\comment{A szoftver átadásával, telepítésével, üzembe helyezésével kapcsolatos követelmények}

% Azonosító, Leírás, Ellenőrzés, Prioritás, Forrás, Komment
\begin{longtable}{| l | l | l | l | l | l |}
\hline
\textbf{Azonosító}   & \textbf{Leírás} & \textbf{Ellenőrzés} & \textbf{Prioritás} & \textbf{Forrás} & \textbf{Komment} \tabularnewline
\hline\hline
... & ... & ... & ... & ... & ... \tabularnewline
\hline
\end{longtable}

\subsection{Egyéb nem funkcionális követelmények}
\comment{A biztonsággal, hordozhatósággal, megbízhatósággal, tesztelhetőséggel, a felhasználóval kapcsolatos követelmények}

% Azonosító, Leírás, Ellenőrzés, Prioritás, Forrás, Komment
\begin{longtable}{| l | l | l | l | l | l |}
\hline
\textbf{Azonosító}   & \textbf{Leírás} & \textbf{Ellenőrzés} & \textbf{Prioritás} & \textbf{Forrás} & \textbf{Komment} \tabularnewline
\hline\hline
... & ... & ... & ... & ... & ... \tabularnewline
\hline
\end{longtable}


\section{Lényeges use-case-ek}
\comment{A 2.3.1-ben felsorolt követelmények közül az alapvető és fontos követelményekhez tartozó használati esetek megadása az alábbi táblázatos formában.}
\subsection{Use-case leírások}

\comment{Minden use-case-hez külön}

\usecase{...}{...}{...}{...}

\usecase{...}{...}{...}{...}