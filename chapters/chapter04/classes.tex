\subsection{Cell}
\begin{itemize}
	\item Felelősség\\
	Absztrakt osztály, a pálya egy celláját reprezentálja és alkalmazza a rálépő roboton az adott cellatípusra jellemző műveletet
	\item Metódusok
	\begin{itemize}
		\item void stepOn(MovingObject) : Absztrakt metódus, a paraméterként kapott objektumra alkalmazza a cellától elvárt műveletet.
	\end{itemize}
\end{itemize}

\subsection{EmptyCell}
\begin{itemize}
	\item Felelősség\\
	Az osztály felelőssége a pálya egy olyan cellájának reprezentációja, ami már nem képezi a robotok számára a pálya részét, azaz a rálépő robotok meghalnak.
	\item Ősosztályok\\
	MovingObject
	\item Metódusok
	\begin{itemize}
		\item void stepOn(MovingObject) : A paraméterként kapott objektumot utasítja, hogy ölje meg magát.
	\end{itemize}
\end{itemize}

\subsection{GameMap}
\begin{itemize}
\item Felelősség\\
Az osztály felelőssége a játék logikájához tartozó fő objektumok karbantartása és a játék léptetése. Ennek megfelelően rendelkezik az összes robottal, illetve a játék térképével. Ebből az osztályból minden játékhoz pontosan egy objektum tartozik.

\item Attribútumok
	\begin{itemize}
		\item cells: Cellákból álló mátrix, gyakorlatilag a térképet tárolja.
		\item rounds: A játékból hátralevő körök száma.
     	\item actors: A játékmenet belső aktorait, azaz a robotokat tárolja.
     	\item toCollide: Egy map adatstruktúra, ami a koordinátákkal indexelve tárolja, hogy az adott helyen melyik robotok állnak. A robotok ütköztetéséhez használjuk. Minden kör végén ürítjük.
     	\item toKill: Egy lista azokról a robotokról, amiket a kör végén meg kell ölni. Minden kör végén ürítjük.
	\end{itemize}
\item Metódusok
	\begin{itemize}
		\item void addActor(MovingObject) : Az actors listához hozzáadja a paraméterként kapott objektumot.
		\item void step() : A körök végén hívódik meg, a felelőssége a játék léptetése. Minden robotot utasít az ugrásra, ha leugrottak a pályáról akkor megöli őket, valamint utasítja az azokat a cellákat, amikre robotok ugrottak, hogy fejtsék ki hatásukat a rájuk ugró robotokra. Ezen kívül érzékeli ha egy cellán ütközés van és utasítja a robotokat a feloldására. Ha elfogytak a körök, leállítja a játékot.
		\item void kill(MovingObject) : Megöli a paraméterként kapott objektumot, azaz eltávolítja a játék logikájának nyilvántartásából (actors lista).
		\item void addCell(Cell, Point) : A térképen második paraméterként kapott helyre elhelyezi az első paraméterként kapott cellát.
		\item void terminate() : Leállítja a játékot és felszabadítja a már nem használt objektumokat.
		\item Cell getCell(Point) : A térképről visszaadja a paraméterként kapott pozíción álló cellát.
		\item GameMap(int x, int y, int r) : Az osztály konstruktora, létrehoz egy x*y méretű pályát, és beállítja a körök számát a harmadik paraméterként kapott r-re.
	\end{itemize}
\end{itemize}

\subsection{MovingObject}
\begin{itemize}
	\item Felelősség\\
	Egy absztrakt osztály, ami képes egy, a pályán mozgó objektumot reprezentálni, és elvégezni ennek a léptetését.
	\item Attribútumok
	\begin{itemize}
		\item pos : A mozgó objektum  pályán való elhelyezkedését tárolja.
		\item velocity : A mozgó objektum sebességét tárolja koordináták formájában.
		\item alive : Bool típusú változó, igaz ha a mozgó objektum életben van, hamis ha már meghalt (azaz kiesett a játékból).
		\item oily : Bool típusú változó, ha igaz az értéke, akkor az azt jelenti hogy a mozgó objektum olajos cellán áll, így a sebessége nem módosítható.
	\end{itemize}
	\item Metódusok
	\begin{itemize}
		\item void updatePos() : Lépteti a mozgó objektumot, azaz hozzáadja a helyzetéhez a sebességét (vektoriálisan).
		\item void collide(List<MovingObject>) : A paraméterként kapott listában szereplő mozgó objektumokat ütközteti egymással. (A listában önmagának is benne kell lennie.) Ha van a cella szomszédságában elegendő olyan cella amin nem áll robot, akkor mindegyik ütköző robotot áthelyezi egy ilyenre, különben megöli őket.
		\item Point getPos() : Visszaadja a mozgó objektum jelenlegi pozícióját a pályán.
		\item void setPos(Point) : A paraméterként kapott helyre helyezi a mozgó objektumot, azaz beállítja a pos változót.
		\item Point getVelocity() : Visszaadja a mozgó objektum aktuális sebességét.
		\item bool isAlive() : Igazzal tér vissza ha a mozgó objektum életben (játékban) van, különben hamissal.
		\item void die() : Megöli a mozgó objektumot.
		\item void setOily() : Beálltja az oily attribútum értékét igazra, azaz megtiltja a sebességvektor módosítását az adott körben.
		\item void halfVelocity() : Megfelezi a mozgó objektum sebességét, ragacsos cellánál használandó.
	\end{itemize}
\end{itemize}

\subsection{OilyCell}
\begin{itemize}
	\item Felelősség\\
	Az osztály felelőssége a pálya egy olajfoltos cellájának reprezentációja és a rálépő robot számára annak megtiltása, hogy a sebességét módosítsák.
	\item Ősosztályok\\
	MovingObject
	\item Metódusok
	\begin{itemize}
		\item void stepOn(MovingObject) : A paraméterként kapott objektumot utasítja, hogy tiltsa le a sebességvektor módosításának lehetőségét.
	\end{itemize}
\end{itemize}

\subsection{PlainCell}
\begin{itemize}
	\item Felelősség\\
	Az osztály felelőssége a pálya egy folt nélküli cellájának reprezentációja.
	\item Ősosztályok\\
	MovingObject
	\item Metódusok
	\begin{itemize}
		\item void stepOn(MovingObject) : Egy folt nélküli celláról lévén szó, nem tesz semmit a paraméterként kapott objektummal.
	\end{itemize}
\end{itemize}

\subsection{Point}
\begin{itemize}
	\item Felelősség\\
	Az osztály egy koordinátapárt képes tárolni, és ezen műveleteket végezni.
	\item Attribútumok
	\begin{itemize}
		\item x: A pont X koordinátája.
		\item y: A pont Y koordinátája.
	\end{itemize}
	\item Metódusok
	\begin{itemize}
		\item Point(int x, int y): Konstruktor, paraméterként a pont X és Y koordinátáit kapja meg és állítja be a megfelelő attribútumnak.
		\item int getX(): A tárolt pont X koordinátáját adja vissza.
		\item int getY(): A tárolt pont Y koordinátáját adja vissza.
		\item void half(): A tárolt pont mindkét koordinátáját a felére csökkenti (alsó egész részt véve).
	\end{itemize}
\end{itemize}

\subsection{Robot}
\begin{itemize}
	\item Felelősség\\
	Az osztály felelőssége egy robot reprezentációja, és a robot cselekedeteinek irányíthatóságának biztosítása kifele.
	\item Ősosztályok\\
	MovingObject
	\item Attribútumok
	\begin{itemize}
		\item distance : A robot által eddig megtett távolságot tárolja.
		\item oilyNum : A robot rendelkezésére álló olajos foltok számát tárolja.
		\item stickyNum : A robot rendelkezésére álló ragacsos foltok számát tárolja.
		\item map : Statikus attribútum, az adott játékmenethez biztosít hozzáférést.
	\end{itemize}
	\item Metódusok
	\begin{itemize}
		\item int getOilyNum() : Visszaadja a robot rendelkezésére álló olajos foltok számát.
		\item int getStickyNum() : Visszaadja a robot rendelkezésére álló ragacsos foltok számát.
		\item void addVelocity(Point) : A paraméterként kapott sebességvektorral megnöveli a robot jelenlegi sebességét.
		\item void placeSticky() : A robot a jelenlegi pozícióján elhelyez egy ragacsfoltot.
		\item void placeOily() : A robot a jelenlegi pozícióján elhelyez egy olajfoltot.
		\item Robot(int oily, int sticky) : Az osztály konstruktora, inicializálja a rendelkezésre álló foltok számát a paraméterek alapján, valamint 0-ra állítja a robot sebességét és az általa megtett távolságot. Ezen kívül élőre állítja a robot állapotát és engedélyezi a sebesség módosítását.
		\item void setMap(GameMap) : Beállítja a map attribútumot a paraméterként kapott értékre, azaz hozzáférést ad a robotnak az adott játékmenethez.
		\item void collide(List<MovingObject>) : A paraméterként kapott mozgó objektumokkal ütközteti a robotot a MovingObject osztályban leírt módon. Fontos, hogy paraméterként önmagát is megkapja.
		\item void die() : Megöli az adott robotot, azaz eltávolíttatja magát az aktuális játékmenetből.
	\end{itemize}
\end{itemize}

\subsection{StickyCell}
\begin{itemize}
	\item Felelősség\\
	Az osztály felelőssége a pálya egy ragacsfoltos cellájának reprezentációja és a rálépő robot sebességének megfelezése.
	\item Ősosztályok\\
	MovingObject
	\item Metódusok
	\begin{itemize}
		\item void stepOn(MovingObject) : A paraméterként kapott objektumot utasítja, hogy felezze meg a sebességét.
	\end{itemize}
\end{itemize}
